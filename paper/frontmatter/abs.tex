\newpage
\TOCadd{Abstract}

\noindent \textbf{Supervisory Committee}
\tpbreak
\panel

\begin{center}
\textbf{ABSTRACT}
\end{center}

%This document is a possible Latex framework for a thesis or dissertation at UVic. It should work in the Windows, Mac and Unix environments. The content is based on the experience of one supervisor and graduate advisor. It explains the organization that can help write a thesis, especially in a scientific environment where the research contains experimental results as well. There is no claim that this is the \textit{best} or \textit{only} way to structure such a document. Yet in the majority of cases it serves extremely well as a sound basis which can be customized according to the requirements of the members of the supervisory committee and the topic of  research. Additionally some examples on using \LaTeX are included as a bonus for beginners.

Technical fields such as computer science and software engineering have placed an emphasis on collaboration and teamwork, and training students entering these fields is a challenge that educators and researchers have attempted to tackle. To assist students in developing teamwork and communication skills, courses have integrated learning activities where students make contributions towards each other's learning. However, traditional learning tools such as Learning Management Systems are limited in their capabilities to support activities involving contributions.

This thesis explores the use of GitHub, an essential tool in the software development community, as a collaborative platform for computer science and software engineering education. GitHub provides users multiple opportunities to collaborate with each other and contribute to each other's work, opportunities that may be extended to education. We investigated how educators use GitHub's unique features, such as its pull requests and transparency features, to support their course activities, as well as the benefits and challenges that emerge from GitHub's use from both the instructors' and the students' perspectives. Findings from this work contributed to our understanding of the viability and effectiveness of GitHub as an educational tool as well as demonstrate what benefits GitHub provides beyond traditional learning tools. As a result, we provide a set of recommendations for instructors wishing to use GitHub to augment their courses that would utilize the tool's features to improve educational activities such as supporting student contributions to course materials and providing detailed feedback to students.
