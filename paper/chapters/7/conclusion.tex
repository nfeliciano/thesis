\chapter{Future Work, \& Conclusions}

This chapter outlines the possibilities for future work and makes concluding statements regarding GitHub use in Education.

\section{Future Work}
From these case studies, a number of questions and possibilities for future work was raised. This section outlines these possibilities.

\subsection{How would the use of this tool affect student performance?}
Because of the exploratory nature of the work, I sought to obtain teacher and student perspectives regarding just the viability of GitHub as a tool for education. However, other studies have investigated using tools such as wikis \cite{minocha2007collaborative} and how they possibly affect or correlate with student performance. This seems to be the natural extension of this work: running a field experiment to see whether or not using the tool effectively simply engages the students more or if it can ultimately affect grades.

\subsection{Building a tool more focused on education}
Discussed in the last chapter, this is another area that can be explored in the future. One could use an open-source tool like GitLab, which shares most of the same features as GitHub, and transform it into a tool more suitable for education by adding some privacy options and some administrative features. From there, one would test the tool in a similar study to identify its strengths and weaknesses, and whether a tool that is more focused on educational activities would be well-received.

\subsection{How scalable is the use of this tool in a course? What considerations have to be made for bigger class sizes?}
Another question that remains unanswered from this work is how this use of GitHub for the courses might scale. With class sizes of over 100 students, for example, would using GitHub work for earlier computer science courses? Haaranen and Lehtinen \cite{haaranen2015teaching} used GitLab in a study with 200 students and found it effective regardless. However, it would be interesting to see which findings and benefits remain, and what new challenges emerge from larger class sizes.

\section{Concluding Statements}
This thesis explored learning tools in computer science and software engineering education, exploring GitHub as a solution to the problems that traditional learning systems suffer from such as a lack of focus on student contributions and a `walled garden' approach to education where content is unavailable to outside communities. As contributing to code and to the community has become an important part of being a software developer, the learning tools in this field need to account for the need to develop related skills such as teamwork, critical analysis, and communication.

In speaking to early adopters who used GitHub for educational purposes, we uncovered how GitHub could be used for these purposes and what benefits and challenges accompany those uses. For instructors, they are given novel ways to allow their students to participate in class, and they are provided a more transparent way to grade projects and assignments on GitHub. However, with no shared knowledge base of suggested practices (when the study was conducted), it's difficult to determine exactly how GitHub can be most effectively used in a course.

I then conducted a case study to explore the student perspective and uncover what benefits they might see from using the tool or what limitations they may experience. For students, the ability to provide and receive feedback from each other was an important benefit, alongside the experience and training they received in using a tool relevant to their careers outside of the university. Privacy concerns, however, was an important issue for some students, where some were not comfortable having their work be publicly available.

Overall, this work suggests that using GitHub is viable in an educational context, despite the fact that GitHub is not designed for education. From the findings of this work, I believe that `The GitHub Way' of working is worthy of consideration in the development of future learning tools, particularly for tools designed for computer science and software engineering courses. The ability to collaborate with others and contribute to the course and to other students' work provides valuable experience for students in those fields.
