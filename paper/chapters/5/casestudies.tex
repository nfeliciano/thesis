%No argument - what did I actually find? What is the impact?

%Bring quotes up a level - tie back to other resarch (put in discussion?)

%Poor literacy! Organize better.

%Improve / Refine themes

%Tool related benefits/challenges vs. Teaching/learning related challenges

%Why did I do semi-structured interviews?

%Limitations!

%Diagram of findings


\chapter{The Student Perspective}

After having gained insight on why and how educators use GitHub to augment their classes, as well as the benefits and drawbacks of using such a system from the instructor perspective. From the interviews conducted with various educators, the instructors explained multiple benefits such as the ability to better monitor student work and the ease in which they can reuse and remix course material from other instructors. However, they noted other benefits which impact their students; benefits such as learning how to use a tool relevant in their field or the ability to make changes to the course material. With these potential benefits to students, it becomes important to determine what other benefits using a system like GitHub might have for students, as well as discovering what challenges they may experience or concerns they may have with using GitHub for their courses. It is important to explore the student perspective to determine the viability of GitHub and `The GitHub Way' for education.

\todo{check this paragraph}
As such, I chose to conduct a study where GitHub was used to support the teaching and learning activities in order to explore how students perceived GitHub as a learning tool. The instructor would utilize GitHub in similar ways to the uses described by instructors in the previous chapter - to host course material, to host projects and assignments, and to have discussions. In this study, my questions were exploratory, in order to learn how students felt about GitHub's effectiveness as a platform for education and for their coursework. It was important to place a focus on the students as learning tools become more geared towards student participation and contributions. From exploring the student perspective, we can determine how GitHub as a tool might help enable a participatory culture \cite{jenkins2009confronting} or serve as a tool to meet the needs of `The Contributing Student' model \cite{hamer2008contributing}.

\todo{impact of research?}

\section{Research Questions}
The research questions addressed during this study include:
\bigskip
\textbf{RQ1: What are computer science and software engineering student perceptions on the benefits of using GitHub for their courses?} We've seen evidence that the use of GitHub can be beneficial in a number of ways for educators. The natural progression was to see student perceptions on how this tool and this way of working might be beneficial for them.

\bigskip
\textbf{RQ2: Will students face challenges related to the use of GitHub in their courses? If so, what are these challenges?} When adopting a new tool for a course, particularly one not tailored towards education, there may be some friction involved from a lack of educationally-focused features in the new tool. We aimed to identify these challenges so as to make recommendations towards designing a system more suitable for educational purposes.

\bigskip
\textbf{RQ3: What are student recommendations towards how instructors can use GitHub in a course?} Just as there are multiple ways to use GitHub for development purposes, an educator has multiple options regarding how they could utilize GitHub as a tool for their course. We aimed to learn what students considered to be important for instructors when deciding a GitHub workflow that students would enjoy and deem appropriate for their courses.
%open-ended courses
%define a workflow
%use the features

\bigskip
\textbf{RQ4: How do students feel that GitHub as a tool compares to more traditional Learning Management Systems?} Specifically, we aimed to gain insights on student perceptions on currently used educational tools such as Learning Management Systems like Coursespaces (Moodle) and Connex (Sakai) and GitHub's potential as such a portal for student interactions with the course.

\section{Research Design}
Consistent with the study conducted with the instructors, we approached this study with a qualitative approach. As Creswell \cite{creswell2013research} suggests, a qualitative, and therefore exploratory, approach best suits research when a concept or phenomenon requires more understanding because there's little pre-existing research. Moreover, continuing with a qualitative approach to be consistent with the previous study is important because the aim was to investigate similar outcomes from the same phenomenon, but from another perspective. %This also continues the constructivist approach with which the research has been conducted.

Yin \cite{yin2013case} introduces case studies as ``an empirical inquiry that investigates a contemporary phenomenon within its real-life context, especially when the boundaries between phenomenon and context are not clearly evident''. Case study design, according to Yin, should be used when (a) the study seeks to answer `how' or `why' things happen; (b) the study is focused on the natural behavior of participants; (c) the context is important for the study; or (d) there are no clear descriptions of what is happening between the phenomenon and context. Most of these reasons apply to the nature of the research questions asked in this study, and as a result, the case study design was chosen for this work. Specifically, the study was exploratory, as it served as an early investigation on the phenomenon of GitHub in the classroom and to potentially build new theories or derive new hypotheses \cite{easterbrook2008selecting}.

\todo{rewrite}
Specific to software engineering, Runeson \cite{runeson2012case} defines case studies as ``an empirical enquiry that draws on multiple sources of evidence to investigate one instance (or a small number of instances) of a contemporary software engineering phenomenon within its real-life context, especially when the boundary between phenomenon and context cannot be clearly specified.'' In this work, I aimed to draw from muliple sources of evidence - multiple students and instructors - to investigate the phenomenon of the `GitHub Way' approach to classes. As well, in this instance, it was difficult to separate this phenomenon from the context of educational tools in the classroom. It's important to learn student perspectives on the context as well, so as to explore the effectiveness of introducing a tool otherwise not built for the classroom.

\subsection{Recruitment}
This study involved the main stakeholders in a course as participants: the professor, lab instructor, and students. The learning tools used in the course directly involves and impacts these stakeholders, and as such, answering our research question must involve getting their perspective and opinions on said tools. I wanted to get their perspectives while the course was ongoing, so that they may recall recent experiences and provided their opinions on GitHub as a learning tool based off those experiences.

For this study, we were opportunistic in finding cases, seeking instructors who could and would try using GitHub, a tool not often used in university classes, for the course. We recruited a professor who wanted to try the tool, and as she taught multiple courses, it set up the study for a multiple-case design. Having multiple cases, it would allow us to explore any possible differences between the two cases, with the goal being to replicate findings across cases \cite{yin2013case}. The two course were Distributed Systems (DS), a Computer Science course that consisted of both undergraduate and graduate students, and Software Evolution (SE), a Software Engineering course that only consisted of undergraduate students. Both classes were similar in size (30-40 students) and in learning activities (two projects, labs every week).

When the term began, I attended one of the early lectures to describe the study and to recruit students for future data collection. They would sign up with their names and emails with the knowledge that participation will be voluntary and that I would only contact them if they signed up to participate. As the course was wrapping up, I attended one more lecture to recruit some more participants in the same manner. This method of recruitment meant that different students participated in the different methods - for example, those who responded to the preliminary survey may not have been interviewed.

\subsection{Research Methods}
Data collection began with a preliminary survey to the students asking them to discuss their thoughts on learning tools in general and on GitHub as a learning tool. This was distributed to all students who signed up to participate, and received 6 respondents from the DS course and 9 respondents from the SE course.

As the main means of collecting data, I would interview the participants who were happy to discuss their experiences and opinions. Interviews began midway through the semester and would continue throughout. Most interviews with the students were one-to-one; however, due to scheduling reasons, some students requested to be interviewed as a group of 2 or 3. Interviews with the students lasted 20-30 minutes and were all conducted face-to-face in a meeting room. Audio from every interview was recorded with participant consent, and I would take notes on the participant's responses for reference. The interviews were semi-structured based on 10 guiding questions and I could probe further with additional questions as deemed appropriate. This supported the exploratory nature of the work, and allows for discovery of interesting insights.
%citation for semi-structured interviews; runeson?

The instructors were also interviewed. The course instructor was interviewed 3 times, once at the start, once in the middle of the semester, and once after the course concluded. This was useful in guaging her perspective throughout regarding the degree in which she found using GitHub useful and effective for her needs, as well as to see what she struggled with. As well, lab instructors (one for each course) were interviewed towards the end of the course, to find out firstly how they utilized GitHub in their labs and secondly to discern their opinions on its effectiveness towards the learning activities they engaged the students in. Interviews with the instructors had a similar format to those with the students: semi-structured, 20-30 minutes long, with around 10 guiding questions.

%follow-up survey

\subsection{Data Analysis}
The analysis of the interview data followed a similar pattern to the study described in the last chapter. Every interview was transcribed carefully, then read and re-read for familiarity, noting important sections or responses. The next step was to code the data by labeling them into various segments based on the research questions of the study. Afterwards, I identified themes and concepts that surfaced multiple times. After separating the themes into well-defined categories, a list of themes was compiled.

\todo{add table of themes here?}

\section{How GitHub was Used}
The professor opted to use GitHub in the same way for both courses, using its features in three pivotal ways: material dissemination through the course repository, lab work through the `Issues' feature, and project hosting through various repositories. Advanced uses from instructors we spoke to in Chapter 4, such as utilizing Pull Requests and Assignment Submissions, were not used for these courses. The instructor was aware of some of these features; however, as a novice user of Git and GitHub, she didn't feel confident using it beyond her knowledge.

The main use case was material dissemination: the professor hosted a repository which all students could access to find the work they had to do for any given week. The instructor would update this repository weekly, adding in lab assignments, links to readings, and the student homework for the week. All of this would be organized into a calendar table made from Markdown, and visible right on the home page of the course repository as a readme.
%screenshot

The other main use was in the \'Issues\' page of the repositories, where all labs (2-3 hour long sessions once a week separate from the course lectures) were hosted. These labs would often involve either using a tool and reporting results or student project work and giving feedback in some form to each other. A dedicated issue would be posted for each lab, similar to a forum post, and students would then make comments on these issues based on their lab work.
%Lab assignments, however, were somewhat different between the two courses, where for Course #2 (Distributed Systems), labs tended to have more complex instructions

The last use was for project hosting. Although GitHub use was not mandatory, most projects were hosted on GitHub and individual repositories. These repositories would be public, so others in the course (and outside, in some courses), can view the work and give feedback. %listed on Coursespaces

\section{First Questionnaire}
After recruitment, a questionnaire was sent out to all students who indicated their willingness to participate during the recruitment stage. The purpose of this questionnaire was to determine the general level of experience and familiarity with GitHub as well as to collect early impressions on GitHub as a learning tool. The questionnaire received 9 responses from the Software Evolution (SE) course and 6 responses from the Distributed Systems (DS) course. The questions are listed in Appendix x.

\todo{add q's to appendix}

Students who answered the questionnaire were generally familiar with GitHub as a tool, with only one student in both courses indicating that they were unfamiliar with the tool. The majority of respondents had not experienced courses where their instructors would use GitHub to manage course activities. When asked about how its use might benefit them, students in the DS course mainly discussed the benefits in using it for collaborating in group projects, while some raised the concern that it may not have much value for the course overall when used similar to an LMS. In the SE course, respondents discussed the real-world experience using the tool would provide alongside the project collaboration advantages, as well as the public nature of projects where their work can be seen and reviewed regularly by other students and the instructors.

When asked about potential challenges, students in both courses had similar concerns, such as the implications of having publicly viewable work on cheating and plaigarism and the lack of threaded discussions. Overall, many of the responses gathered from this questionnaire were reflective of the responses students gave during the interviews.

\section{Participants}
We were able to conduct inteviews with 13 students from SE, and with 7 students from DS, one of which is in both courses. The main distinction between the two courses were that SE was an undergraduate-only Software Engineering (SENG) course whereas DS was a Computer Science course with a mix of undergraduate and graduate students. Otherwise, as outlined above, the courses were laid out in a similar manner. Table z summarizes the students who participated in interviews.

\todo{table of participants}
% \begin{table}[h]
%     \vspace{1pt}
%         \caption{Information on Participants}\label{table:interviews:students}
%     \vspace{1pt}
%     \begin{center}
%         \begin{tabular}{cc}
%             \hline
%             ID & \multicolumn{1}{|c}{Prior GitHub Experience}
%             \hline
%             DS1 &
%         \end{tabular}
%     \end{center}
% \end{table}

%table here
%DS1 - P1
%DS2 - P5
%DS3 - P7
%DS4 - P8
%DS5 - P9
%DS6 - P15 R

%SE1 - P2
%SE2 - P3
%SE3 - P4
%SE4 - P6
%SE5 - P10
%SE6 - P11
%SE7 - P12
%SE8 - P13
%SE9 - P14
%SE10 - P16
%SE11 - P17
%SE12 - P18
%SE13 - P19

\section{Findings}

The research findings are presented in accordance to the research questions of the study. From the analysis of the data, I extracted several common patterns in the responses of the participants as they came up and categorized them into themes. Each theme will be discussed in detail and then accompanied by exemplifying quotes from the interviews as well as a list of participants that echo the theme.

%Beyond the themes, some interviewees gave unique feedback that were not mentioned by others, but were noteworthy for several reasons - for example, they might speak to the potential of the tool or are unique drawbacks that only few people experienced. These will be highlighted appropriately and reasons will be given for why they are noteworthy answers.

%table of themes?

\subsection{RQ1: What are computer science and software engineering student perceptions on the benefits of using GitHub for their courses?}
In this section, I discuss the benefits that emerged from the students' perspectives from the three main uses of GitHub in their courses: for schedule and material dissemination, for discussions, and for hosting their project work. Many of these benefits stem from GitHub being a tool commonly used in industry, as well as from the advantages that Git offers for managing work and group work.

\textbf{Benefit: Getting Introduced to a Tool Relevant for Careers} \\

To preface the first few themes, I reiterate that GitHub as a tool is, in the current landscape of software development, a very popular tool for working collaboratively. As such, it is an essential skill for developers to be familiar with either GitHub or with distributed version control systems in general, particularly when working on collaborative, multi-person projects.

Students came into the course with varying degrees of experience with GitHub, as shown on table x \todo{refer to table}. Some hadn't used it at all, while others were very knowledgeable about the tool either through their own uses, through group projects for other classes, or through co-op jobs. Many, at least those who have done the majority of their undergraduate studies in the University of Victoria, has had some experience with Subversion, a different version control system tool, as Subversion was taught in a second year course.

Many of the interviewees mentioned that the use of the tool in class provided a good introduction to the tool for the students: \textit{``Yeah I think it's pretty good. I mean one thing is that because I'm using it in class, it's made me learn the tool\ldots and that's where the big takeaway is that I've been able to transfer those skills, I've done some other projects just on my own time using GH.''} [P3]

%Others felt that in general, it was a good way of learning that way of working, which is beneficial for their careers: \textit{``I think [learning about the tool will] be very useful. Especially if I wanna work with any, open source, any newer companies, a lot of them have a lot of open source projects. Just knowing about version control and stuff like that is gonna be \ldots really useful in the future.''} [SE4]

For the most part, students who supported this theme believed that the use of GitHub in their course for projects helped them experience a style of collaboration that they will later encounter often in their careers. In comparison to the use of more traditional LMS, one student noted why using GitHub might be advantageous for him: \textit{``Well I like how it's like the extra bonus of more practice of something you're gonna use in industry, whereas none of us are gonna use Coursespaces or Connex when we're out on a co-op or out on a job.''} [SE3]

As well, putting their projects on GitHub provides practice for real-life scenarios. SE8 describes why it was beneficial to have their work publicly available for both classmates and outsiders to see: \textit{``I think when you go and work in software development too, you should get used to [having] lots of eyes being all over your work; that's just the way it's gonna be, so it's a practice before real life.''} [SE8]

Beyond this benefit of using GitHub in programming projects, which is what it was designed for, the basic use of GitHub to manage course activities such as material dissemination and discussion was still beneficial to students as an introduction to the tool. \textit{``It's a good introduction to GitHub as a platform; it might not be a good introduction to Git as a tool. Because there's a lot of wizardry that you can do with Git that you'd never learn just doing what we did here\ldots But definitely a good start to get people using Git.''} [SE11]

Some were introduced to some of the features available on GitHub that they were not necessarily aware of. \textit{``This is the first time I've actually used the issues portion of GitHub. \ldots So it showed me that portion of the capabilities of GitHub.''} [SE13]

Out of all the benefits described by students, this benefit emerged the most, as _{SE2, SE3, SE4, SE5, SE6, SE7, SE8, SE11, SE13, DS4} supported it. The importance of this benefit is underpinned by these students asserting their intention to continue using GitHub or to use GitHub even more after the course ends, whether or not they've had prior experience with GitHub.

%Those who had little experience with using Git and GitHub were given an introduction to the features of the tool and how `The GitHub Way' works, particularly when they used it for their group projects. Others familiar with using Git and GitHub cited others in the class or in their groups that initially weren't familiar with it and acknowledged the way in which the class allowed them to experience this tool that many felt was important to know about as a software developer or as a computer scientist.

%\textit{``Well I think the first thing is not quite everybody had used Git before. So for some people it was a bit of an introduction to it, but I think that's definitely a good thing. Better now than going on a coop, and the first day being like, \'here, here\'s the git repo'.''} [SE3]

% All of the interviewees asserted that they would use GitHub outside the class, whether or not they've had experience with it beforehand. \textit{``I've actually used it more now that since I've started using it in class actually.''} [SE2]

% \todo{candidate for cutting}
% \textbf{Benefit: Using GitHub as Storage} \\
% As an extension to this, many believed that GitHub use in the course benefited them in two primary ways. First, for those who were yet to be acquainted with GitHub, this introduction allowed them to experience its features and benefits, with many of these students asserting their intention to use GitHub more after the course is over. As well, for every student who would use GitHub to host and manage their projects in the course, many enjoyed the fact that their completed projects would be on their GitHub account publicly viewable for anyone to see. When asked about the motivations behind putting their work on GitHub, P8 explains: \textit{``to actually show to people and maybe have some people collaborating or maybe have employers see that I have worked on stuff \ldots I have heard about some of my other friends, like their interviewers actually ask them to look into their repos and they actually download it and install the software and they would talk about it.''} [DS4]

% One student believed it to be an useful tool for storage from the beginning: \textit{``I think it would have been great to have, especially in first year, if you did open one and use it throughout your education, it really does provide a place where all of your stuff is stored. You'll be glad that you can go back to all your old work.''} [DS3]

\textbf{Benefit: Ability to Use GitHub as a Portfolio}
As an extention to the above, many believed that using GitHub to work on and host their course projects would be beneficial for them in the future. A common element amongst students who supported this theme was the concept of putting their code from other courses or from personal projects on their GitHub accounts. As such, many found it useful to store their code on GitHub for various reasons. SE5, for example, organized their code on GitHub for easy access when helping friends: \textit{``I know that when you're trying to help somebody out, you can always just say \'Check out my GitHub\', I know I've done that with a few of my buddies \ldots and I don't have to search through my files, it's just on GitHub, and you look on there. It's a good organization tool.''} [SE5]

A common reasoning many interviewees used to explain this benefit was that GitHub could serve as something akin to a portfolio for students, where their projects and their code hosted publicly could help them when job hunting. It has been acknowledged that many employers now look at GitHub for hiring purposes \footnote{\url{http://www.cnet.com/news/forget-linkedin-companies-turn-to-github-to-find-tech-talent/}}; in fact, some students even had interiewers asking them to show them their GitHub account: \textit{``I think all three companies that I applied to this semester wanted me to link to my GitHub. So I was really lucky that I had [a class] project on there. And I think when this project is done too, it'll also be really nice to have up there, after we clean it up.''} [SE6]

While some students had interviewers expect their code during the interview: \textit{``These days I see that employers also want to see your GitHub page. While I was giving an interview for my coop, he did actually go into my GitHub profile and try to compile some of my code, so they do want you to have some online presence on GitHub. So it does help you in that, and since it's been used so widely, using it is necessary I think.''} [DS2]

%\textit{``Well I believe it's good for future employers. I remember I put directly on my resume saying you can check out the work I've done on GH. I included the link right on there and every person I handed my resume to were just like \'hey, fantastic!\' \ldots it's a good way to get your skillset out there.''} [SE5]

As such, that students were able to use GitHub as a portfolio of sorts where they can show off their projects to future potential employers was, for many an important benefit of using the tool in this manner. There were students who were introduced to GitHub in this course but knew the importance of having work on GitHub, and this benefit motivated some students to continue putting their work on GitHub. For others, having their course projects from these courses hosted on GitHub could serve to benefit them for seeking employment in the future. This benefit was supported by _{SE5, SE6, SE7, SE8, SE11, SE13, DS3, DS4}.

\textbf{Benefit: Letting Students Contribute to the Course} \\ % (changing material, tagging others in discussion, etc.)
In the previous chapter, one of the benefits that using GitHub over traditional learning management systems was the ability to allow students to make changes, fixes, or suggestions to the course material. Traditionally, this could be done by speaking to the instructor either in-person or by email, and the instructor could then make the changes as necessary. With GitHub, however, the students are able to make Pull Requests, changing the material themselves, notifying the instructor and prompting them to `accept' or `close' (reject) their Pull Request (PR) as they deem appropriate.

In the case of these two courses, throughout the semester, only three Pull Requests were made to make fixes to the material. What's more, they were all made in the first month of the course and by only one student who was well-versed in GitHub and in both courses. They explains their reasoning: \textit{``I like being able to fix the mistakes that she might make like with a bad link or something by making a PR \ldots I really like being able to do that because you know, it makes me feel a little more involved.''} [SE1]

However, early complications hindered this type of participation, as the early PRs to fix the material or add links were not merged quickly enough by the instructor and no one else was able to help. Another student described the issues occurred \textit{``because we did not have the access. If we had the access, then I think people would have collaborated \ldots I feel that either we should have had the access to merge it, or at least someone else would have had [access] who would have merged it quite quickly, like someone handling the Pull Request. \ldots [Otherwise], that just defeats the purpose.''} [DS2] This is potentially an issue with how GitHub handles collaborators and who can accept PRs.

SE6 offers the advantages of this system as opposed to the traditional way of suggesting material or fixing mistakes: \textit{``I think everybody's had experience with like mistakes in the course material\ldots The alternative is just emailing the prof and asking them to change something \ldots this is always there, and they can always check it to see if there's something. This way someone can actually make the change, all they'd have to do is accept it.''} [SE6]

Of the three PRs submitted to the courses by the one student, two others participated by either trying to accept the PR sent (and failing) or adding a `+1' to a PR, supporting its acceptance. Many other students agreed with this benefit, as it received support from _{SE1, SE3, SE5, SE6, SE10, SE13, DS2, DS4}. %The benefit seems to stem from the fact that the process is straightforward (the instructor only has to accept or reject the PR) and it gives students agency in making these changes.

% As such, some believed the professor should have given indication that this was an option. \textit{``I think [the idea is] good, but I think it would've needed to have been advertised more that she was looking for input on things, and if she said that, maybe more people would have [contributed] to maybe propose like extensions for assignments or something.''} [SE7]

\textbf{Benefit: Letting Students Contribute to Each Other's Work} \\
In these courses, projects were open and visible to other students, which allowed more opportunities for student contributions. This is demonstrated by a student's group working with others:
\textit{``For instance, one [issue] was our script wasn't taking in command line arguments if there were spaces in them properly. And then someone was like, you can just put in quotes. And we were like \'oh, that's a lot better than what we were doing\'. And then to be able to see what other people are having problems with and give suggestions. Even at one point, they were trying to find refactorings, and we said hey you can use our tool, it'll help.''} [SE3]

Making students host their course projects on GitHub and relying on GitHub heavily in the course resulted in students looking at each other's work (mandatorily or voluntarily) and making contributions in the way of advice or suggestions. As well, students would actually utilize code from other groups and help fix them when necessary. \textit{``I believe that one other group decided for project 2 to use [our project 1] and they made a couple of Pull Requests I think''} [SE10]

As well, two specific lab assignments asked students to look at the repositories of other groups and comment on them, an exercise that students found useful, or at the very least, interesting. For students who spoke in detail about this benefit, they enjoyed receiving comments from the others: \textit{``Yeah I liked getting the comments, I liked knowing that people were kind of checking it out, and I assume they would let me know if I was doing anything horribly wrong, and I didn't get any of those comments, I'm assuming that everything was going alright.''} [SE5]

One student extended this concept to the idea of peer reviewing, where students would judge the work of others and make comments on them. This student explained the benefit of having others looking at and judging their work: \textit{``I thought [peer reviews] was the best way to learn actually\ldots It forced you to put you in a position where you have to defend what you did, which I think is good for quality because you have to actually care.''} [SE11]

Helping other projects through discussion or even code offers students new ways to participate in courses that is unique to GitHub and similar systems, as students are effectively collaborating with each other with the aim of producing better work. _{SE2, SE3, SE5, SE7, SE10, SE11, SE12, SE13} described this benefit.

%Discussions were another way that students felt was a good way of participating in the course. Although discussions are a part of multiple other tools, GitHub's discussions, via the Issues feature, provided an advantage that others didn't - Mentions. By typing someone's username preceded by the \'@\' symbol, the user gets \'mentioned\', and this has two implications: a) other users reading the comment know who the comment was addressed to and b) notify the user being mentioned that someone mentioned them in a comment.

%In the Issues for the classes (which acted as the Labs), many students would use it simply to sign their work or acknowledge their group members as part of their answers. However, during labs where interactions were more encouraged, such as when they were set to comment on each others' work, students would use this Mentions feature, which would notify those mentioned that a comment was directed at them. Some found this useful: \textit{``I really like how once somebody's commented on the thread, you can just use the little @ symbol and send them a notification and vice versa. When someone mentions you in a comment, you'll see it on your email. Which is good.''} [SE2]

\todo{Getting Inspiration from Each Other's Work?}
% \textbf{Benefit: Getting Inspiration from Each Other's Work} \\
%seeing other groups work
% As described above, the open and public nature of how GitHub was utilized in these courses meant that students could easily see each other's projects. Though most only looked at other group's projects when mandated by a lab assignment, some interviewees found this aspect of group work helpful: \textit{``It's nice seeing what everyone else is working on, because you're working on your own project, and you kind of get into your own little world, and then.. Yeah it's really interesting to see the other creative ideas that other people come up with \ldots I actually think there was one person on [another] group, they had a really really interesting question, I kind of followed their repo as they went along, and finally got answer.''} [SE5]

%Some, however, found this exercise only potentially helpful:
%\textit{``I think that it could be useful, I mean for my project, I definitely read through the other comments. But I already like a pretty good idea of what I thought I was gonna do, so I definitely took some of them into consideration, but I wouldn't say it was like super valuable for me, personally. But maybe if you like, if you actually were having big issues with your projects, and you don't know where to start, it might be useful.''} [SE2]

% While others took little time to go through other groups' repositories beyond what was required for the lab. When asked why, SE6 responds: \textit{``Maybe [I'll look at the other repositories] at the end of the course. But like I know what ours looks like right now \ldots It's a work in progress and I assume other people's are the same right now.''} [SE6]

% Some, however, took little time to go through other groups' repositories beyond what was required for the lab assignments, often because of time issues: \textit{``I think it is important [to be able to see other people's projects and give feedback] and it's neat that it's all out in the open \ldots the other thing is there's just so much going on with all the courses that it's hard to pay attention to every other project, and even get remotely invested in them, because it's almost a challenge to make sure yours is working.''} [SE10]
%
% For the lab assignments, every student would post something that was visible to the rest of the class as a response to the week's lab topic. Many students discussed the benefits of having this type of submitted work available publicly. Students found value in being able to see the work done by others for reasons such as comparing their work, learning from others, and even just curiosity.

% \textit{``The one nice thing I like about just being able to do the issues like this as opposed to like a regular assignment submission is that you can see what the other people are doing. Which is really helpful if you're a certain standard of how much everyone else is typing, as well as if you're just out of ideas and you need somewhere to start with you're like, alright, that's an interesting idea, maybe I could look at that a little bit more.''} [SE2]

% \textit{``It's nice being able to see everyone else's answers, it's not just you writing an assignment and you handing it in and you don't really know what anybody else has done. And the questions have been constructed in a way [such as] \'find three tools\', so it's nice being able to review the tools other people have found, because you might not have found them either.''} [SE5]

% It should be noted, however, that this style of lab assignment submission is easily replicated in traditional LMS through the use of forums and discussion boards.

\textbf{Benefit: Keeping Each Other Accountable} \\
One benefit that stemmed from GitHub's transparency features was the ability to see a history of commits to a project. This was cited by some students who used GitHub to manage their group projects, where they could easily see if and when their partners submitted work. Their repositories have an account of when each change was made, which gives collaborators an easy way to track the work being done on the project. This helped them keep up with each other's work, and is a commonly cited benefit of using systems like GitHub \cite{dabbish2012social}: \textit{``You can see exactly what the other person has contributed, and you can look it up again a month later. \ldots So then if they're trying to say that `I did this huge massive thing', and you look and it's only like teeny-tiny, then it's a good way to keep accountable. And it's good for yourself too, because you know they can see your work, so you wanna make sure that it's top notch and easily readable''} [SE5]

%\textit{``And just being able to see what other people are doing on the repo you're working in, is kind of a motivation in a way too. It's like, oh hey they're working on this, I wanna do something cool like that too.''} [SE3]

Indeed, this helped them keep themselves and each other accountable, so they know exactly what work each member of their group put in: \textit{``we decided to switch to Pull Requests instead of just committing straight to master, because.. for a couple of reasons, first of all, if there's something like majorly wrong with it, everyone can see it right? And the second thing is, everyone sees it, so if people have to work on [the same code], in the future, which we all did, then they know exactly what just went in, so that next time they come to the code and pull it, they're not like `where did this all come from?' ''} [SE9]

This is a useful feature to have when working in group work as it allows for awareness between group members. By using GitHub for their group projects, students were able to take advantage of the collaborative features that GitHub offers to improve their process or their product. This benefit emerged from _{SE5, SE9, SE11}.

%add SE3 quote here from slide 16

%professors too?


%able to see when repo is updated
%Having the course materials and schedule on the GitHub repository was helpful to some students thanks to the Transparency features of GitHub. If a change has been made or if the professor added something new to the repository, they would be informed in their News Feed and, if they are subscribed to the repository, by an email sent to them.

%\textit{``I think once you star something, you get notified for every push to master, so you get all the changes to like, this home page readme, and anytime a new issue is posted.''} [SE7]

%Unfortunately, some experienced issues with this, particularly when they were subscribed to the repository. This will be covered later when their responses pertaining to Research Question 2 are highlighted. As well, only a few students cited this as a benefit, with some not even knowing that it was possible to get notifications by watching the repository and describing that as a weakness of the system.

\textbf{Benefit: Version Controlled Assignments} \\
% Though not necessarily related to how GitHub was used in their class, many students described advantages to using GitHub to host their school work, including their projects for this class. For one, students enjoyed being able to do their course work from any computer. Some students who were more experienced with using GitHub claimed they would use it for all of their course work for this very reason. \textit{``And that's sort of where we're at, it's like I use it for everything. Even when there's no specific mandate for it, just the ability to pull it down in multiple places is worth having.''} [DS3]

% \textit{``For me, the reason I started using it is because you know, I would start working on a project at home, and then realize 15 minutes into it that this was my saved version from a week ago, and I had done work since. Like \'oh it\'s on the school computer I need to go get it'. [And then] it finally clicked \ldots I think it's really beneficial, by the time you get to 3rd year, for like networks, the big multi-week coding assignments that it\'d be really useful for.''} [SE6]

%version control, potential feedback. p7, group
Using version control for their assignments and projects benefited the students in multiple ways. P1, who worked alone on their project, recounts that using GitHub \textit{``makes it more traceable.. you can see when and where [your work has been done]''} [DS1] This student utilized the history of their commits to remind themselves where to pick up from when they work on their project. For others, the ability to revert to previous states of the code was useful: \textit{``You're working on a project, and you make a change that breaks everything, well you can just go back to a different commit, one that works. Boom, fixed, try again.''} [SE11]

Although the instructors for these courses did not use their repositories for marking, some students believed the system could allow instructors to give constructive feedback as they build their projects and assignments. One student believed that the ability to see the student's process could be important: \textit{``You'd see all the mistakes they made getting there too, which is just as important to learning as the finished product.''} [DS3]

Another described a hypothetical situation where professors could use student's repositories as submissions as opposed to the traditional way of submitting through LMS - sending the code only when finished. This student said that this way of submission would be \textit{``so much more useful \ldots You could see everybody's contributions, you could comment on them too \ldots Unless you're doing a live code demo with a TA or any instructor, you're not getting any real feedback [with the traditional submission system] \ldots You have no idea where you lost the marks or where you went wrong.''} [SE8]

In all, using Git to manage their work and hosting it on GitHub were beneficial for the students because of the ability to pull from anywhere and to see and revert to previous versions of their work. As well, students believed that having instructors use version control to mark their work could be beneficial if implemented, because of the new ways in which they could provide feedback. _{SE1, SE2, SE8, SE11, DS3} contributed to this benefit.

\textbf{Benefit: Connecting with the Outside} \\
Finally, the last benefit pertains to the way in which work put in GitHub is often publicly available for others to contribute to. In one xample, one student is highly active in the community of a certain programming language. For their first project, they were building something highly related to the language and they advertised the work to the community, who would go on to help his work in multiple ways:

\textit{``So here I have people involved in the discussion, these are just people in the community I've been talking to about how to do different things, and they've been giving me suggestions. And that's really cool because I actually have like some community involvement in my course project.''} [SE1] He notes how this is helpful for him: \textit{``But for me, I find it really validating when someone else is like \'that\'s really cool, have you considered doing this?' ''} [SE1]

This was the only student interviewed who utilized the public nature of the course projects to solicit outside contributions. However, the exposure to GitHub gave students opportunities to discover work outside of the course and use repositories to aid their projects. When prompted, most interviewees mentioned that they sought out public repositories either to pull their code and use them or to find inspiration for their own projects. One student recounts an experience where their group looked at an open-source library: \textit{``we just looked at how Gitstats, [an open-source library] did it, and then wrote our own thing into our project \ldots I think that more than anything is the biggest reason why Git should be used for education, because it takes I think until you start being forced to do it\ldots to actually go and look at other people's code, and I think looking at other people's code is like the most important thing.''} [SE6]

Speculatively, this likely would have happened regardless of whether or not GitHub was pushed by the professor as students tend to seek other code and libraries for their projects: \textit{``And in industry, the first thing you do is check Stack Overflow, look for someone else who has done the same thing and jack their code.''} [SE7] However, this speaks to the advantages of GitHub when students are able to pull in outside work to affect their own assignments and projects. _{SE1, SE2, SE3, SE4, SE6, SE7, SE10, SE12, SE13, DS2, DS5} mentioned looking at outside work and public repositories for their projects.

\subsection{RQ2: What are the challenges for students related to the use of GitHub in their classes?}
This section outlines the challenges the students described relating to GitHub use in courses. \todo{more}

\textbf{Challenge: Privacy is All-or-Nothing} \\
While the public nature of GitHub provided several benefits, others acknowledged the potential issues that follows - namely that it may not be appropriate for a class environment. One student put it eloquently: \textit{``So [using GitHub for your work has] got benefits and drawbacks: benefits being that other people can access your data, drawbacks being that other people can access your data.''} [SE4] While students hosting their work publicly may have advantages, they come with a cost, particularly in instances when work should be more private.

Most interviewees didn't mind that the class repository and their project work was completely public. However, many could see the potential problems that might surface because of public hosting. First, students mentioned that although they would ideally put 100\% effort into all their submissions, including lab assignments and comments, this is not always realistic due to the time crunch students face. DS3, for example, noted that although it can be advantageous to put their code up in that employers are able to see their projects, it might backfire if the employer has a different mindset. \textit{``I think it comes back to what do you want to show your employers? When your employer looks at your work, will they understand that work I submitted in Git was when I didn't understand yet what I was doing, I was still learning? \ldots If I could make the assumption that an employer would understand that, I would have no problem with it being public. That said, I can't make that assumption. I have to assume that everything they look at they're judging in the harshest light possible. So I try to show only things that are of quality that I'm proud of. And that's unfortunately not a lot of the classwork until I'm done with it and the final product I'm happy to show, but all those steps getting there, they're often filled with pitfalls and horrible programming and badly factored code.''} [DS3] As such, courses potentially mandating the use of Git and GitHub could be problematic for students if employers are looking at work-in-progress in a negative light.

%not everything is 100% effort
Although one or two students felt that they would only submit work they were proud of, many acknowledged that sometimes, students rush through their work, and they therefore might not want work of that nature to be public. \textit{``You know it would actually be nice if they were separate or private somehow so I wouldn't have to go through everything and sanitize all the stuff I've submitted, because you know, for as much as you'd want to think you're putting 100\% into it, you're not really, you know, writing some great work of art or careful analysis, so private would be nicer. For things like that.''} [SE6]

%not everything is of interest to public
As well, some felt that some of the work done on the course repository wouldn't even be of interest to the public or to potential employers, and as such they saw no need for the repository to be public. Regarding splitting up their comments between GitHub (where they're publicly viewable) or Coursepsaces (where only the class can see them), SE1 asserted that \textit{``I'd rather have them be private. But only because there's not a whole lot of participation, so I don't feel they're of interest to someone publicly.''} [SE1]

% \textit{``It's good for finished projects, but not all that in-between stuff. I don't wanna show these issue things that I've submitted, to employers because they'd have to wade through that. They don't wanna see that, they wanna see finished projects and what that looks like after you're done.''} [SE6]

Yet, others saw no issue, and even preferred all their work to be in the public space. \textit{``Personally I don't have a problem with it being public. I would like to have a good online activity of myself on GitHub, so that's not really an issue. I'm not really concerned if someone is going to read my blog or not.''} [DS2]

There are workarounds to some of these privacy issues, where students do not have to attach their names to the work they put into the GitHub repositories involved in the class. \textit{``I think part of that would be.. you can decide that on your own, depending on if you use like your main git account or just make like a separate git account for your class and whatnot.''} [SE3] Indeed, one student did create a new GitHub account just for contributions to the class. Unfortunately, this student did not want to be interviewed, and group members were uncertain as to what the motivation behind creating a new user was; presumably, they were motivated by some of the issues posted above.

In summary of this theme, although many students enjoyed the benefits that came with making their work publicly available, many also acknowledged that those benefits are accompanied by a number of caveats. Importantly, some students described the lack of a middle ground, where in the context of this course, students had to make all their work publicly available. This challenge may be mitigated by the instructor giving the students the option of creating a new GitHub user account for work done in their courses, as suggested by SE3 above. This challenge was acknowledged by _{SE1, SE4, SE5, SE6, SE7, SE10, SE13, DS3, DS6}, while _{SE3, DS2, DS5} disagreed.

%first group's submissions rule all
% A benefit cited earlier was the openness of the tool, where students are able to see each other's work and submissions to the lab assignments. However, many acknowledged that this way of working where everybody in the class can see submissions simply could not work in other types of courses, particularly those in which assignments have only one solution. \textit{``In this particular course, because each person's project is so different and unique, there's really no way you could do plagiarism, it just doesn't make sense. But if for some courses where everybody is giving the same exact assignment, and you're all on GH, I think I'd be a little more nervous about putting my code up. ''} [SE2] This is further discussed in RQ3.

% In fact, some saw this effect even on the lab submissions in this course, even though lab work typically varies from group to group. \textit{``I'm not sure if it's good or not that we can see other people's submissions as we go along in the lab. Because I think, because of that, all of the submissions have ended up looking like the first submission.''} [SE6] However, as mentioned by many students, this is potentially a benefit, by giving them examples to follow or a standard to measure up to.

\textbf{Challenge: Lack of Training on Git and GitHub} \\
Another issue that many students described with using GitHub in these courses is education and training - there were wildly varying degrees of experience with and knowledge of GitHub and it's features, which presented difficulties with using it in the course. For example, one student asserted that many of the benefits, such as the ability to make Pull Requests on the class material, were unknown to those less experienced with the tool, and as such could not be taken advantage of. If the professor did not set a precedent for that behavior, it would not be used. \textit{``I think you just have to, 1. advertise it so that the students know [to] use this as a communication tool. And then 2., kind of layout, or give some examples on how it could be used, I guess.''} [SE9]

This lends itself to a bigger issue - educating the students on Git and GitHub, as well as on the instructor's intended workflow for using the tool in the course. As the course instructor for these two cases was more or less a novice at using GitHub, this made it difficult to educate the students and caused some frustration for some of the interviewees. In fact, most students who were asked mentioned that the course could have benefited from more education on Git, GitHub and what they can do with it. Many said that they could have hosted a lecture or a lab dedicated to learning the tool, perhaps at the beginning of the course or as an extra session. \textit{``I think it would've been good to do some demo \ldots cause I think [the instructor] talked too much about theory in class and there's no actual coding or no actual demoing.''} [DS1]

%\textit{``I wish [the instructor] knew how to use GH, that would be nice. Because yeah I was submitting pull requests to actually make her readmes that were .markdowns markdown files instead of html files, and stuff like that. ''} [SE1]

%Those experienced with GitHub still acknowledged the usefulness of hosting such a session: \textit{``there's people who have never used it, like [a group member] in our group has never used it so she's just getting used to all the features \ldots if we wanted to maybe tie it all together, it probably would've been useful to have a little Git session for the people who didn't know how to use Git''} [SE6]

%\textit{``Even just like on the first day of the lab just going through some of this stuff. I think what we did on the first day was that if you didn't have a GH account, you were supposed to go and create one. That part's really straightforward, commenting on issues is really straightforward, that's no problem. The only problem is when you actually start working with code and working with versioning...''} [P3]

One student acknowledged the potential difficulties in hosting such a session: \textit{``On the other hand, when someone teaches it to you, it often doesn't make sense until you actually do it yourself. Cause you'd actually have to go through the struggles of actually doing a commit and like pressing all the buttons, so I don't really know how much could be done in that regard.''} [SE2]

Interestingly, students also asserted that there should be a bigger emphasis on teaching version control systems such as GitHub in the undergraduate level at University of Victoria. As it stands, there is one required course that students said touches on distributed version control systems and how to use them, heavily utilizing Subversion and touching on Git. Some, however, did not feel that it was enough, particularly when SVN is much less popular. \textit{``I think in [SENG265], we did SVN, which is a good introduction to the idea. But I don't think it's widely used anymore. ''} [SE3]

% \textit{``, I could tell you I think it was SENG 265. And it was, we were kind of introduced, we were told what version control was, and \'oh yeah you could use SVN or you could use Git, or both\'. We were kind of introduced to it, we were never required, or even really pushed to use it''} [SE5]

% Some would lament the theoretical focus they believed the undergraduate software engineering or computer science departments had: \textit{``I think that's one of the biggest problems, even SENG being taught in a university that focuses a lot on theory, and not a lot on technical.. like one of our only classes where we got to actually learn technical skills, and it's a really technical area.''} [SE8]

Some of the students believe that students should be getting an account pretty quickly after their first introduction courses to Computer Science. \textit{``if I was teaching someone how to code, as soon as they start working on code that was bigger than 100 lines, I would teach them how to use version control.''} [SE1]

% \textit{``The first class where you have a big project, like a group project where you're sharing code, it should be taught''} [SE6]

It is important, however, to communicate the benefits of using such a system. \textit{``The first challenge is to get them really interested, because many of them are exposed to many other tools already. So I think to really bring up the advantages [that] GH can bring to the students so they can use GH as a primary tool and convince them that it's really good''} [DS5]

Of course, this issue is related to tool literacy, where a more experienced instructor might have been able to better educate their students on GitHub and the features they intended to use. Beyond the instructor, a greater focus on DVCS and what students can do with them earlier in the curriculum could solve the issues presented above where students were not able to utilize the benefits of GitHub due to inexperience and unfamiliarity. This was supported by _{SE1, SE2, SE3, SE4, SE5, SE6, SE9, SE11, SE12, SE13, DS1, DS3, DS5}.

\textbf{Challenge: Notification Overload} \\
Although few students brought up this issue, it remains a challenge because of how GitHub handles notifications from the repository. As it stands, the only way to get notifications from a course repository is to `Watch' the repository. `Watching' provides two different options: 1) to get a notification and an email only when the user is mentioned in issues or commits and in discussions the user has commented on, or 2) to get a notification and an email when anything at all happens in the main branch (master), someone makes a comment on issues, commits, or Pull Requests, and when someone makes or accepts a Pull Request. The `Watch' feature came with some drawbacks, not the least of which was when a student did not know about this feature:

\textit{``I didn't like that [repository] at all, because I didn't get notified when she adds stuff to there, so I don't really know what's going on without remembering to check it on GH. ''} [SE9] This student did hear about the `Watch' solution, but thought that it \textit{``would be a good solution, but it might be overkill. For like a spelling change.''} [SE9]

The main issue is that unless students are `Watching' the repository, they would not receive email notifications for any activities unless they were directly mentioned. However, if they do `Watch' the repository, they would receive an influx of notifications for every user comment on the discussions, which is too noisy for some. This was an issue experienced by a student who was engaged less in the activities of others because of this noise: \textit{``It sent me a million emails, both of [the tools] actually. I should have just turned that off, but I was worried about missing something. Because everytime someone would post, you would get another email \ldots I actually did not read anyone else's feedback because it was just so many emails, to be totally honest.''} [SE10]

As such, the `Watch' feature was problematic for a course like these, where every single comment would trigger a notification and an email and cause an overlaod of notifications. Responses from _{SE7, SE9, SE10, SE11} contributed to this theme.

%- Integration with UVic
%Another point that students felt would hinder GitHub as a learning tool was the lack of integration with the University of Victoria login, similar to that which LMS such as CourseSpaces have. This means that students will have to create new accounts to an external site (if they didn't have one already) and, more importantly, that the administrative tasks that LMS usually take care of such as grades and class participant lists cannot be done on a tool like GitHub.

%Hmm.. The one thing about CS is that it's all integrated with the current school system. So in terms of like adding your grades and feedback that way, I don't think profs are even allowed to upload grades into GH, there'd be some big issues there. *laughs*. So.. that'd be one thing. [P3]

%Uh, it's fine, but we like to, we like to get a more into interactive with the CS. For example, if somebody post something on GH, we couldn't know anything that is not on CS, because these two are not connected. [P9]

%Right now I have to say GH is... to me, I value them equivalently, just because the university enforce CS into their system. So I'm pretty sure in the future if the university can include GitHub into their university system, I think students are more prefer GH. [P9]

%Uh, the only thing I can think of is maybe somehow.. I don't even know how it would work, but if they could somehow be.. connected with UVic. Like, I'm not even sure how that would work, but.. some sort of academic affiliation or something. [P10]


\subsection{RQ3: What are student recommendations towards how instructors can use GitHub in a course?}

Given that the use of GitHub in these two cases were relatively basic, many students, particularly those who are experienced with using GitHub for collaboration purposes, had ideas on how GitHub could be further utilized to be more beneficial for both themselves and for their professors. Many students brought up topics such as which classes GitHub could best serve and the need to utilize more features of GitHub than their course did. This section outlines those responses, highlighting what students suggested were important when considering the workflow for using GitHub in a course.

\textbf{Recommendation: Use in More Open-Ended Courses} \\
As discussed earlier, students had concerns regarding the public nature of the work they host on GitHub. While most students interviewed did not mind their work and their comments being in the public space, there were concerns regarding how this way of working could apply to different types of courses, where students are afforded less freedom in the nature of their work. %As well, some did not even see sense in keeping a course repository open for the public because of issues highlighted earlier, such as that it just won't be interesting for outsiders and that their work might often be rushed and therefore unappealing to have on display.

As such, some students, _{SE1, SE5, SE6}, would suggest that for the workflow done on this course, where the discussions are self-contained, the course repository did not need to be public. This would avoid some of those issues discussed for students, but would similarly conflict with some of the benefits extracted from our instructor interviews in the last chapter. A suggestion for avoiding these issues came from one interviewee: \textit{``I think that as long as we have the option to make [our discussion comments] private, maybe after the course ends. So keep it intact while the course is ongoing and then we have the option [to change the privacy], everything will be okay.''} [SE12] Currently, GitHub does not support doing such tasks, unless the course instructor decides to privatize the course repository as a whole after a course ends or an individual student deletes their comments and posts. %While this is a potential solution, these tools like GitHub would need work to meet such a solution. %this might be DSCS-worthy

However, students had opinions regarding what type of course would best suit using GitHub. Interviewees suggested that a course similar to these two cases, where the work is very open-ended and could therefore exist in a public space, is where using a tool like GitHub would most benefit the students. When asked about their experiences with seeing others' projects in this course versus in other courses, P12 said: \textit{``I would say this class is specifically different because we had so much flexibility over what we were doing. It's not like in our Operating Systems class, [where] it's like go make this shell that does this, this, and this. Where this was way more open ended, everyone's doing something different, so even if you could see what everyone else is doing, no one could've helped us.''} [SE7]

Students would acknowledge that the open-ended nature of this course was what allowed the use of GitHub in these courses to be successful, but that it would not work in less open-ended courses. Regarding the potential use of GitHub in their future courses, P10 discussed their thoughts: \textit{``Like I said, I like seeing other people's work and whatnot. Maybe not if everyone has the same assignment, because everyone's just gonna cheat off each other.''} [SE5]

%\textit{``It works really well for this too, because it's not the same thing that everybody's submitting. So even if the format's the same, the content's gonna be different. So obviously if it was something where there was an answer, it'd be awful for it.''} [SE6]

% \textit{``With something like Operating Systems class, where everyone's building a shell, and it's all in C, you all have the same system calls and it's more like a puzzle where you gotta put it together the right way, you can just look at someone else's and be like `oh that's how that's done', right? \ldots If they have everyone's code up available to everybody, then that's kind of pointless.''} [SE7]

These students related back to the privacy issue, where having completely public work might be a detriment to the work being done when there are concerns of plaigarism. As a result, the prevailing belief was that using GitHub would work well in courses where the work was open-ended and would not work well in courses where the opposite is true. Students believed that instructors would have to consider the nature of the work before deciding on the workflow they use GitHub with, or indeed, whether they want to use GitHub at all. This consideration was revealed through responses from _{SE2, SE5, SE6, SE7, SE13}.

It should be noted that there are ways to use GitHub privately within a course, where even student assignments are private. This type of workflow involves the instructor creating an organization and having each student create private repositories for their work, so as to keep them private from each other. However, this workflow would still minimize many of the benefits listed in RQ1.

% With no formal assignment submission feature, students believed that GitHub would struggle when students were uploading their assignments if all the assignments had a specific answer. SE6 brainstormed a potential workaround: \textit{``I was thinking you could even do that with like pull requests, you just start a PR, each person just a PR, and the teacher just accepts all of them on the due date, or whatever.''} [SE6] However, this is indeed a workaround and may require more effort from the parts of both the instructor and the student.

\textbf{Recommendation: Mandate the Use of GitHub Features} \\
The students more experienced with GitHub mentioned that the more collaborative features of GitHub should have been further utilized so as to take advantage of the uniqueness of using GitHub over traditional LMSes. One issue that some students discussed was that they saw little reason to use GitHub for courses if it was in the way these courses did it for material dissemination. As mentioned in RQ1 above, for example, only 3 Pull Requests were made throughout the semester for a variety of reasons.

DS3 was very outspoken on why using GitHub for this course was somewhat unnecessary. \textit{``I don't see any benefit that GH has offered that we wouldn't have had in CS. All it appears to me is it's a place where it's a file repo, \ldots and we already have that.''} [DS3] In elaborating, they note that there's potential there, but the unidirectional nature of the work being done meant that there was little benefits provided. \textit{``If there was a way to collaborate on the material, that would be useful \ldots But in this class, every one of our labs so far has been demo to the lab TA, so nothing's going back to GitHub \ldots Maybe if we were submitting things to it, maybe that would be helpful. I can see how it could be useful, it's just that in our usage it's not really adding anything to the experience.''} [DS3] It should be noted that this student's project group did not use GitHub to collaborate, using Docker Hub \footnote{\url{https://hub.docker.com/account/signup/}} instead. That said, they were very critical of why GitHub was used as the course repository.

SE7 echoes these sentiments: \textit{``I think that you can accomplish the same thing with a simple HTML website, honestly \ldots It's not using a lot of the features of Git, like looking at changes, commits, Pull Requests. The issues were kinda cool for the lab, and, again, you can accomplish that with any sort of forum, I would think \ldots We're not actually delivering code to the professor, so maybe it doesn't make a ton of sense [to be using GitHub].''} [SE7]

As such, these students, as well as a few others, believed that GitHub was not being used to it's full potential in their course by not utilizing its full feature set. The underlying suggestion was to consider which features of GitHub the instructor would like to use, such as Pull Requests or grading via commits, and use those features thoroughly. As it stands, some of the benefits they described to using such a system were only possibilities. An example, which will be highlighted later, was in the Distributed Systems course, where even the issues were not used during labs for discussion. \textit{``So basically we had to show it to our TA that we have done [the lab], and he used to mark it in a piece of paper. So putting it here [in the issues] was not really necessary right?''} [DS2]

An important lesson to learn is that GitHub only equips instructors and students with the possibility to take advantage of the benefits on offer. _{SE3, SE5, SE6, SE7, SE11, DS2, DS3, DS4, DS6} described this theme.

\textbf{Recommendation: Define and Advertise a workflow} \\
Students acknowledged that GitHub was not being used to its full potential and that there was a lot of confusion surrounding the use of two tools (GitHub and Coursespaces) to cover up the drawbacks of GitHub. Students, however, tended to being displeased with this decision, as the split caused some confusion: \textit{``One thing I really don't like is that we have both systems set up, and so sometimes the announcements are in GitHub, and some of the times, they're in CourseSpaces, and that can get kind of confusing, like did she post an assignment here or here?''} [SE2]

This was an almost uninamous issue between the students interviewed, with only a few stating that they did not mind either way. Most mentioned that they would have preferred it to be all on one tool, even if it was all public in GitHub. As a result, many students suggested that it would have been important to define a workflow for using this tool in a course in order to gain the benefits described earlier in the chapter. This workflow could include aforementioned activities such as utilizing Pull Requests or using just one tool instead of two. In the case of Pull Requests, for example, students advocated that the instructor should be advertising their use, thereby defining to the students that contributing to the material would be part of the course workflow.

\textit{``I think [the idea is] good, but I think it would've needed to have been advertised more that she was looking for input on things, and if she said that, maybe more people would have [contributed] to maybe propose like extensions for assignments or something.''} [SE7]

One student would assert that although GitHub doesn't do everything needed in a course, defining a workflow will cover up many of those weaknesses. \textit{``Even if there are no enhancements on GH, but if you define a proper workflow of using it, then it can be quite successful, because even the present Learning Management Systems are not perfect right?''} [DS2]

While most students didn't have suggestions as to what workflow to use, they acknowledged the importance of defining it and teaching it to the students early on in the course. SE6 wanted to \textit{``enforce more actual Git and GitHub features in the way that we interact with the course material, and enforced GitHub use for actual projects. In a way that everybody had sort of a base level of understanding. So maybe at the beginning of the course \ldots there should definitely be a time when you learn Git.''} [SE6]

In summary, many of the students interviewed were frustrated at the lack of a cleary defined workflow, and believed that the course could have improved heavily if one had been defined and advertised in the beginning. This theme emerged from interviews with _{SE1, SE2, SE3, SE4, SE5, SE7, SE9, SE11, SE12, SE13, DS2, DS3, DS4, DS5}.

% Beyond the workflow for the course as a whole, some also had difficulties getting on the same page as their group because there was no defined workflow. \textit{``I initially wanted to post all the issues and our conversation so we could account for who's doing what. But they basically didn't respond to them at all, so I moved to email. They responded to that a couple of days later so that was nice.''} [SE1] As such, students believed that having a properly defined workflow would go a long way towards reaping some of the benefits and bypassing some of the limitations behind using GitHub as a learning tool.

%doesn't belong here
% Other students would acknowledge this is a nice benefit of using GitHub in this manner, claiming that it's something they could see themselves using had they known that it was an option. When asked about changes to make with the course and its workflow, SE3 answers: \textit{``Like maybe if we can incorporate PRs in one way or another. Maybe with the idea I mentioned before, of having everyone contributing to the main repo \ldots That would be neat and beneficial.''} [SE3] %\textit{``So it's good that other people, like students can also contribute to it \ldots [it brings] more involvement from other people as well.''} [DS4]

%Uh, I would create students directories, one for each student for the course, they have their own directories. Later on, if there's group work or group projects, I would at least like students to form a group and create their own project directories in there. And they can view their stuff as well as outsiders can also view what's going on the course. [specific workflow]

%Yeah, I would like to see Yvonne maybe using the Wiki a bit more. Because the readme's good, but it's very long. So and the readme, I think should be more of a directory, like go here for this, go here for this, and key updates, like don't forget the midterm on Thursday. But the wiki you can actually construct it like a book. So that would be nice if that was set up. [workflow]

%I think more teaching on maybe more common industry practices when you go about it. Cause there's so many ways you can use GH, and I was actually looking at some other people in our class' GH pages, and the way they did it was quite different from the way we did it. Like we did the whole PR workflow, which, I don't think I'm biased but I think that's like a really good workflow. And then you see like other people aren't even using issues so.. I think maybe just a little more teaching on the capabilities of GH.


\subsection{RQ4: How do students feel that GitHub as a tool compares to more traditional Learning Management Systems?}
One of the goals of this research study was to see how GitHub would fare as a tool that served similar functions to learning management systems (LMS). Instructors that we spoke to in Chapter 4 described using GitHub in such a manner, which made this use case worthy of investigation. The primary purposes of traditional LMS include material dissemination, evaluating students, and discussion among the students and instructors.

\textbf{Comparison: No Threaded Discussions in GitHub} \\
A feature that came up often in interviews when GitHub was being compared to traditional LMS at the University of Victoria (Connex/Sakai, Coursespaces/Moodle) were discussions and forums. For both courses, the students had two main platforms where they could discuss and comment - on the course repository's `Issues' area and on Coursespaces' Forums. The professor separated these by assigning all lab work and discussion to GitHub and the assignment readings and comments to Coursespaces.

Students were generally receptive to the use of issues as posts, particularly as it offered some flexibility otherwise not seen on regular forums. For one, students liked the ability to `Mention' others and be notified when they themselves are `Mentioned', as there is a convenience to receiving this notification. \textit{``I really like how once somebody's commented on the thread, you can just use the little @ symbol and send them a notification and vice versa. When someone mentions you in a comment, you'll see it on your email. Which is good.''} [SE2] This is a feature generally not available in the discussions features of learning management systems used in the University of Victoria.

However, the GitHub Issues way of discussion did not resonate with all students. The main difference between the Issues discussions and those seen in forums in traditional LMS is the lack of `tree-style' discussion in GitHub, where all comments on an issue are linearly arranged by time. In the Coursespaces forums, for example, there are top-level comments in a thread which users can reply to, and those would be arranged in a manner which makes conversation easy to follow. \textit{``GitHub unfortunately doesn't have any sort of like tree style view of conversation, it's just linear, so it's really hard to actually have a conversation between multiple people \ldots As opposed to Coursespaces, which is kind of more tree-styled.''} [SE1]

The main issue with this, according to this student, is following these conversations: \textit{``And you literally have to go and search for all of the conversations from foo and bar and try to piece them together. It would be nice if it had like a reply that would put it right underneath.''} [SE1]

% Students who offered up suggestions for possible improvements to GitHub often discussed changes to this style of commenting, advocating a style similar to that of Coursespaces. \textit{``If there was a way to actually directly reply and keep track of what's going on, it would also be a lot easier to look at your post and see if anyone's replied to it, as opposed to reading every single post and seeing \'oh that mentioned me!\'''} [SE3] There were similar sentiments by P5 and P7.

One student did, however, think there was usefulness in the linear style of GitHub comments as opposed to the tree-style of Coursespaces. When asked if they read through the posts on Coursespaces, P12 responded that they didn't as much as they did the GitHub comments, because \textit{``you'd have to scroll through all the responses before getting down to the bottom to write your own.''} [SE7] %It should be noted, however, that these are two different types of comments on each tool, and that the comments on GitHub were posted more in real-time, so this could have contributed to this disconnect.

% Ultimately, Issues offer some flexibility in that when student projects are hosted on GitHub, students are able to help each other out in each others repositories by filing issues in the project repositories themselves, which is another point of convenience as it consolidates it into the project repository. Of our interviewees, this behavior was seen in only one project, P16 and P2's project, where P4 could not build their code properly and helped them through the process.

Ultimately, however, most students that would discuss compare between the two types of discussion wanted at least the option to have tree-styled comments on GitHub. This was echoed by _{SE1, SE3, DS2, DS3, DS4}.

\textbf{Comparison: GitHub is Not Built for Education} \\
A few students described one drawback behind using GitHub for education - it's simply not built for it. Those that mentioned this particular drawback acknowledged that there's workarounds for many of the tasks needed, but that GitHub certainly struggles to meet some educational needs such as Gradebooks and proper Assignment Submission. In traditional LMS, for example, assignment submission tends to be a private upload of relevant files visible only to the instructor. This simplifies the process, ensuring that there is only one submission.

These courses did not utilize GitHub for any form of submission of their code for projects. However, when asked about GitHub serving as an LMS, some students highlighted this lack of formal submission features in GitHub. As highlighted earlier, for example, SE6 stated that \textit{``If there was gonna be assignment submissions, you'd have to figure out a way to do that properly''} [SE6], and suggested potential workarounds by using Pull Requests. When listing features that would hinder GitHub from being a learning tool, others, such as _{DS3, DS4, SE7}, also highlighted the lack of formal submission as a potential hindrance.

The potential benefits of using GitHub were also minimized because it is not a tool geared for educational purposes. As discussed in the last chapter, GitHub does not handle file types common to class material well, such as PDF Documents and Powerpoint Presentations. This makes it difficult for students to make changes to course material that involve these file formats. \textit{``I think one disadvantage of.. or one drawback of GH is that you cannot actually see the diff [of] commonly used files such as PPTs or PDFs, so you can't really use it for correcting professor's slides, or PDFs.''} [SE12]

This may hinder the potential of using Pull Requests to contribute to the material, as it complicates the process. SE13 was dissuaded because of this complication: \textit{``I think for readme files, it's a lot easier to edit, cause you can edit directly in GitHub. But for other files, you'll probably have to change and make a branch and then commit it and then send a PR, it might actually be more work.''} [SE13]

Students would also bring up the inability to do tasks common in most LMS. For example, for the University of Victoria and many other universities, including a gradebook is impossible to do without a tool in which the data is controlled by the university. \textit{``Because at least with Coursespaces, you can access your grades and access your schedule that's coming up, and you can access your entire transcript. But that needs to stay in the school domain, not in the public GitHub.''} [SE5] As such, there's difficulty managing the more administrative tasks because of GitHub being public and data being stored on GitHub's servers rather than the university's.

% Overall, students would acknowledge that GitHub was, at the moment, not geared towards education, which makes it difficult to do some tasks, making it necessary to find potentially clunky workarounds. \textit{``But I believe that mainly as GitHub is for projects hosting, it's not specifically for learning, like Coursespaces or some other dedicated tools \ldots For example, if we want to initiate some discussions, there's a workaround way to use issues, but I think issues are [typically] for projects and stuff. [P8]''} [DS4]

SE9 acknowledges the difficulties in using GitHub in a context it wasn't designed for: \textit{``Right now it definitely feels like we're using a tool that's geared for something else and trying to throw education on top of it \ldots So if you could integrate them better into another tool or maybe a plugin for GitHub if they ever did something like that.''} [SE9] Unfortunately, the two potential solutions they described are not realistic options at the moment. When asked exactly what GitHub would need to provide to make it more suitable for education, they noted some features typically found in LMS: \textit{``Like deliverables, like grades, announcements that you'd actually get by email, not just code, commit changes.''} [SE9]

Because GitHub is a tool designed for developers, some educational activities are not reasonable to use GitHub for because the features don't exist. While there are workarounds to some of these missing features, they require extra effort on the part of the instructors and the students. This theme surfaced from comments from _{SE2, SE5, SE6, SE7, SE9, SE11, SE12, SE13, DS4}.

\textbf{Comparison: GitHub's Simpler Interface} \\
Many of the students discussed their preference towards GitHub interface over the Coursespaces interface. For some, this was a result of their familiarity with GitHub, while others simply enjoyed the way in which the readme was laid out. Some students, such as _{SE8, SE12}, complained that the User Interface of the learning management systems used in UVic often did not meet their expectations. \textit{``I think Connex and Coursespaces were made specifically for course management. So I think they're more capable [for that]; just functionality wise, the capability's there, but maybe the UI-wise, is not as friendly.''} [SE12]

_{DS1, DS4, DS5, SE13} described the GitHub User Interface as cleaner or friendlier, particularly in the way the course schedule has been laid out in the front page readme. _{SE1, SE13} would assert that they were more comfortable with the GitHub interface because of their familiarity with the tool, as opposed to a tool like Coursespaces. \textit{``Also, because it's how big GitHub is and how familiar I am with it, I can navigate it a lot easier than say, Coursespaces.''} [SE13]

%- Offers more opportunities for engagement - not really a theme.
%Some students described GitHub as having offered more opportunities for participation and for engagement than traditional LMS, and could therefore see it as beneficial in that manner.

\section{Validation Survey}
I conducted a validation survey to determine the accuracy of the findings that emerged from the interviews with the students. The survey was distributed during the final lab of the course, where students were asked to fill in a 5-10 minute survey about their experiences. 18 students responded from the DS course (4 of which were interviewed), while the SE survey received 15 responses (9 of which were interviewed). I note some of the interesting and relevant responses below.

In both courses, students indicated that their level of familiarity with GitHub was raised from when the course began. Students strongly agreed that they would continue using GitHub for group work and for individual work. Although many of these students could have already been using GitHub for these purposes, the number of students who strongly agreed suggests that they were able to gain insight on the benefits using GitHub could provide for them.

As expected, the SE students tended to agree more with feeling more involved in the class from viewing and commenting on other projects. This followed a similar result to whether or not students having their own work commented on or their questions being answered was useful, where the DS students would be more neutral. As well, most students in SE felt that there was enough collaboration or student contribution to justify using GitHub in the course, whereas it was more balanced in DS, where only half the respondents agreed.

Surprisingly, most students in both courses disagreed or were neutral with the idea that their school work should not be available in public. In DS, more students disliked the discussion system on GitHub as compared to Coursespaces. Both courses also tended to disagree that the classes needed a tutorial in the beginning of the semester, and both strongly agreed that Git, GitHub, and other DVCS should play a bigger role in UVic Education.

\section{Instructors' Perspectives}
In interviewing all three instructors (one course and two lab instructors) after the course wrapped up, they were able to give their perspective regarding these cases and on some of the themes extracted from student interviews. The interview questions asked are listed in \todo{Appendix}. Overall, they were satisfied with the use of the tool and optimistic about its potential for future use. There was, however, a key difference between both cases as seen in the talks with the lab instructors, and this was the main challenge highlighted by the course instructor throughout the semester. In the Distributed Systems course, lab assignments were done on their own and then shown to the lab instructor. This contrasts to how lab assignments were handled in the Software Evolution course, where students would post a comment to the corresponding lab `Issue' with their work and who they worked with. As such, one course ended up using the tool much more than the other, and this is explained by the lab instructor below.

\subsection{Course Instructor}
The course instructor overall felt that their relative inexperience with the tool hindered some of its use, an issue a few students brought up. They did, however, speak of the importance of working with a tool that has real world relevancy, as they believe it was what drove the successful use of GitHub for course projects. \textit{``For students, my sense is, for those who are ready to get in the game, it really amplified their commitment and really was in keeping with the real world \ldots So I believe that the benefit was that kind of participation in something that's bigger than just what we can normally give them.''}

% Moreover, they felt the importance of giving students experience with this real-world tool. \textit{``I would say, I believe [this experience with GitHub is] a really necessary thing for us to be giving our students right now. If our students can't handle this kind of environment when they're going out to do co-op or, their careers, we're doing them a disservice.''}

Indeed, they believed that this ability to work in an ecosystem that they would need to work in for their future careers really helped drive the enthusiasm towards the course. In fact, they asserted that even if the repository was not publicly visible, relevancy would have offered enough of a reason for students to participate. \textit{``I actually think it could've been private too and still there would've been some more excitement with just sharing with each other. But I also feel like being a part of something that's going on around them, you know, contributing to a space that they know is a part of their ecosystem, I think that's it. So even if it was a private GitHub, I think it would've upped the game.''}

%However, they thought that the public nature of the repository had benefits as well. When asked about the Pull Requests, they believed it was a good way of allowing students to contribute to the material: \textit{``I mean ultimately, of course they can post [corrections and changes] on Coursespaces too, and in a sense it accomplishes the same thing in terms of contributing to the body of knowledge in the class. But there's something about giving them the exercise of doing it more publicly I think.''}

They also saw the advantages to the openness afforded by having students collaborate using GitHub for their projects, as it allowed the teaching team to track student contributions to projects and raise concerns based on that activity. \textit{``Certainly, when it came to looking at the projects and people that participated in projects, it was very interesting. And there was a difference between what we saw in terms of the code repo and how students were participating in their own repos and some of the things that they were handing in. \ldots I think for us, these subtleties are important.''}
This echoes the benefits described by instructors in the last chapter.

% As described earlier, the teaching team ran into an issue later when there was a disconnect between the course instructor's criteria for marking and how one lab instructor ended up marking. Although she was unsure of the main contributing factor towards this disconnect, she felt there was a possibility that the issue of being unable to put grades on a public space like GitHub may have been a contributor. \textit{``I think that I didn't do a good job of integrating the vision with one of my TAs. And somehow, the effort to support creativity and the public nature of things translated into a very unfortunate way of trying to do marking \ldots Maybe it was in part because [the work] was public and the marking had to be private. There was this strange disconnect.''}
%explain this better?

As described earlier, the teaching team ran into an issue later when there was a disconnect between the course instructor's criteria for marking and how one lab instructor ended up marking. Although they were unsure of the main contributing factor towards this disconnect, they felt that this problem could have been avoided had they defined a vision of the workflow early on and agreed with the teaching assistants on this workflow. However, they felt it important to gather ideas from students regarding this workflow: \textit{``Because although I didn't get direct pushback, they were saying, `well it would be really nice if the teaching team knew how to use it', and I said right away, `yup I'm gonna be learning'. But at the same time, I think the challenge is to tease out from [the students] what they see as being the right way to use that.''}

% The multiple tools issue that many students had, however, was less of an issue for the course instructor. Their reasoning is explained: \textit{``I'm actually not against using multiple tools, I actually think that's okay, because the world is going to be about using multiple tools.''}

Regarding the idea that an open system like this may not work for courses where there are assignments with a single solution, the course instructor disagreed. However, they did provide a relevant experience in which having publicly available assignments would displease some students: \textit{``I did something before with another platform where it was public and students were doing labs, and of course\ldots the first lab that ran through certainly provided things that other students used. Which I didn't think was a bad thing, students didn't complain to me directly; indirectly, I heard that students in the first lab thought that it wasn't fair \ldots I think it was interesting because it created a hierarchy within the class.''}

They explain that more thought would need to be put into a system like GitHub for a course like with single-answer assignments, but the benefits could be there: \textit{``I just feel like the more we can help them help each other, the better off we'd be. So I'd have to think about how to do [a course like that] more, but I would not be against it.''}

Finally, the course instructor discussed how GitHub matches up to more traditional LMS, explaining how important it is to introduce a system like GitHub to the education of Computer Scientists and Software Engineers. She discusses the concept of the `Ivory Tower' as something that traditional LMS fall victim to, where a course exists in a vacuum. \textit{``I just feel like [GitHub has] got that real world edge, and that is everything to destroy this kind of ivory tower, `I'm just doing what I have to do to get by', horrible thing that sometimes happens to our students, [where] they feel like there's nothing they could do within this system that would really make a difference anyway. Because from an instructor's perspective, I need to get information out to them, and I could do that in an email, in a listserve, I don't know. I need to get them to engage with the material and with each other. And I need to get them to engage with the community in a broader sense. And I think that's critical, and that's what GitHub has.''}

%workflow
%So I think it's almost uh.. if I'm not available, I've gotta make sure that the teaching team can be available to be able to update..

%As mentioned above, the main issue this instructor ran into regarded the difference between the two courses, where one set of students were much more active in participating in GitHub while the other barely participated beyond their project groups.

\subsection{Lab Instructors}
The lab instructors generally agreed with many of the sentiments given by students and the course instructor with their experience in the labs. However, speaking to them highlighted the need to define a specific workflow that encourages the use of the tool in order to see the benefits it could give. Regarding the difference between the two labs, where the two had different levels of participation, one had a defined workflow where the other did not, where the Teaching Assistant for Software Evolution (TA SE) would use the tool much more than the other (TA DS).

\textit{``For me, [the use of Issues] was not applicable, cause \ldots everything was done in the lab, right? So demos were all done in the lab\ldots But if there were more hours allocated for it, I guess students would have posted more.''} [TA DS]

\textit{``Basically I told them to post their responses to [the lab] as a response to that issue. For me it was nice because one, I could see they were doing stuff as they posted things, so occasionally I would get up and I would like to see what they were doing, so I would tell them to post a response even in the middle of the lab just so I could see what they were working on. Just to see in real time, and then I could analyze it, and if I had feedback, go around to them and give them feedback on what they've done so far.''} [TA SE]

It seemed as if the lab instructor for the Software Evolution course wanted to take advantage of the tools available and defined a workflow specific to it, and, possibly as a result, the course instructor felt more enthusiasm from that group of students regarding the course. According to the both lab instructors, the course instructor gave no specific instructions regarding how to use GitHub throughout the course, which may have contributed to the lack of a defined workflow in one case. The Distributed Systems lab instructor would have changed this in future iterations of using GitHub in courses: \textit{``You just come up with a system that works and just stick with it for the rest of the class.''} [TA DS]

As well, they both seemed to agree on the potential of GitHub in terms of benefiting the education of students, as well as how they mark. They enjoyed the openness of GitHub for being able to see the projects as they go on, and then for marking those projects. \textit{``Towards the end of the [each project] run, I would start looking at the projects through GitHub, and then try to provide some feedback to the students before they were due, to help them figure out what they could improve on \ldots the labs were not interactive for me, so it was a good way to provide feedback because nobody was really asking me questions.''} [TA SE]

The potential benefits of the Pull Request system for course material was not lost on the two instructors as well. \textit{``It's a collaborative environment, so I don't see why students would not be allowed to participate in it. Especially in a university environment, people like to restrict things and like put boundaries. But that usually ends up restricting people's creativity. So the more open, for my perspective, the better \ldots I think that if people are able to modify material that will just enhance the class experience for everyone.''} [TA DS]

Regarding the issue of publicly viewable assignments, the Software Evolution lab instructor described his experiences in a course where there was one-solution assignments. Although they acknowledge that it would be more difficult to use a tool like GitHub in less open-ended courses, they felt it could have helped their experience. \textit{``This is nice because I can see the code quite easily. I guess for this class, I wasn't judging them based on their code, I wasn't really looking at the code much. But for [a different course], something like this would be really nice, because I would have access to their code, it'd be a little bit easier \ldots I never had to look at their code anyway other than going over \ldots and checking when they had questions. But in that case, I may have looked at it more had I had GitHub rather than trying to figure out how to download the submissions through Connex.''} [TA SE]

Overall, both felt that with some changes to workflow and maybe to the system itself, GitHub has the potential to to serve as a powerful learning platform, particularly for classes where there are lots of collaboration.

% Distributed
%\textit{``I've not really done any of that [markdown] before but I just thought it was a good way to like.. instead of just having a readme file, just like being able to still write ASCII and being able to format stuff is really cool.''}

%[on looking at projects] \textit{``Just for marking really. But if somebody's putting stuff up, sometimes you just get on there and start browsing. Just start looking at ?.. and what else they are interested in. Because some people have other projects that are on GH right?''}

%[no specific instructions on using GH before the course]

%[on looking at Coursespaces] \textit{``The only thing I used CS for was entering marks.''}

%\textit{``I thought it was alright.. and like it could have been more.. uh.. in-depth, like we could have done more with it. Like most of the stuff that we used was basically just as a CMS. But uh.. I thought there would have been more potential if we had thought about how to use it.''}

%\textit{``I wish there was more ways to say, customize GH. So like sometimes you'd like to have a function and that's just unavailable. But you have the same problem with CS or other systems like that. But I just think that CS for example is like a proprietary system which kind of develops slowly. And I think GH would be in like a position to be better than CS, but then there's like this whole issue about privacy, where this information is stored, etc. So I don't have a solution for it but I don't think CS is the best way to do it, because it's just such a static system that evolves so slowly. And there's always like a limited group of people using it, instead of the evolution rate on that system it's just way slower than everything else.''}

%\textit{``I was skeptical at the beginning, but I think it's pretty decent, like you can do a lot. Especially just having a framework where you can easily write up documentation and make it available.''}

%Software Evo
%\textit{``The only thing.. it wasn't necessarily difficult, but some forums have where you can reply directly to a post. GitHub doesn't necessarily do that... you can't necessarily have it tied to a specific post, so you'd have to provide some context if you want to provide feedback... [my feedback] was mostly face to face, and had I had that capability, I may have done it more.''}

%[on GH's potential] \textit{``I think it's nice because students could have their own repositories and still link back to the classes. I guess it depends, it's nice for these projects because everything's public, it would be difficult for a class where teachers don't want collaboration, like 116, where everyone has to work on their own. In that case, it makes everything a little bit more difficult because everything by default is really public on GH... But overall it seemed like a really good tool to use especially for a course like this.''}

%\textit{``It seems like GH's been better for collaboration than Connex or anything I've worked with.''}

%\textit{``I think it has good potential. It'd be interesting if GitHub actually thought about it that way and added a few features and maybe have a specific portal, but I think it was nice for classes that are quite collaborative. I think it was nice to use issues, and then students.. they get used to using stuff like that as well which is nice. Maybe if they're not using GH, at least get experience using central repositories and version control.''}

%\textit{``That was the best thing, is I could see in almost real-time what they were doing and provide feedback pretty quickly and pretty seamlessly rather than having to search for submissions and things.''}


\section{Discussion}
The motivation behind this study was to uncover student perceptions on using GitHub as an educational tool by asking them to describe their thoughts and experiences as they experience them. GitHub was used in three main ways: (a) as a place to disseminate material and to host the class schedule, (b) as a place for students to submit their lab assignments and discuss these assignments, and (c) a place where most students interviewed would host their course projects, either collaboratively or alone.

\subsection{A Student-Oriented Learning Tool}
%What does GitHub provide? more opportunities for students to participate and contribute!
At a basic level, using GitHub for education can provide similar functions to those of traditional LMS. As discussed in the last chapter, GitHub has the capabilities of providing many of the common activities found in Malikowski \textit{et al.}'s model of LMS features \cite{malikowski2007model}. As to be discussed in the next chapter, however, accomplishing some of the finer-grain features of traditional LMS, such as a formal assignment submission, requires some workarounds. Even though GitHub can serve a similar purpose to educational tools, it was not built for education.

Where GitHub has the potential to move beyond, however, is in addressing some of the concerns regarding traditional LMS outlined by various authors. Mott \cite{mott2010envisioning} discusses the `walled garden' approach of LMSes, where the content is limited only to those officially enrolled in the course, and that they support administrative functions better than actual teaching and learning activities. García-Peñalvo \cite{garcia2011opening} echoes these concerns, asserting that students need to be placed at the centre of the e-learning process. This could be addressed by allowing students opportunities to participate in the course and connect with and learn from each other. GitHub can provide these opportunities for students to become a part of each others' learning, creating a culture of participation \cite{jenkins2009confronting}.

\textbf{The Contributing Student} \\
GitHub provides opportunities for students to engage with their learning in multiple ways due to the multiple ways in which they can participate. Students are able to openly contribute to the course material by making the changes or additions directly to a course repository. Traditionally, students would have to talk to the instructor or send an email to make corrections or additions. GitHub provides a much more open and direct way for students to contribute to the course material. This plays a key role in Collis and Moonen's concept of a `Contributing Student' \cite{collis2006contributing}, where GitHub provides students the ability to drive their coursework.

%contribute to each other's work
When student assignments and projects are public, GitHub can provide students the opportunity to contribute to other students' learning by easily providing direct feedback to each other's assignments or project work. A number of groups in one of the cases in this study utilized this ability by leaving feedback for other groups when they noted bugs or issues in the code, and students seemed to enjoy this ability see others' work and provide feedback as they see fit. Contributing to other students' work may provide benefits in developing soft skills such as communication and teamwork skills \cite{hamer2006some}. An instructor may also utilize GitHub to provide opportunities for students to peer review each other's work, where students may grade each others' work. This could provide potential benefits such as more reflection for students while working and the development of analysis and evaluation skills \cite{sondergaard2012collaborative}.

However, it is important to note that like any technology, accessing these benefits requires the stakeholders to `buy-in' and use the relevant features of the tool to support this pedagogy. It's possible, for example, that there were different levels of enthusiasm for the tool between the two courses because of the difference between how it was utilized in labs, where the first case (SE) would require students to post often, leading to looking at others' responses, while the other case (DS) would not utilize the tool as much, requiring only a demo to the lab instructor instead.

\textbf{Transparency of Activities} \\
%accountability
In describing the benefits of using GitHub to support their group projects, some students described the transparency of activities as helpful for collaborating with each other. Few of the transparency features of GitHub were mentioned by the students - for example, the News Feed or the Graphs were not discussed in the context of group projects. However, some students acknowledged the importance of seeing a history of work from other group members, describing the history as a good way to hold accountability and to keep current with the work. This is in line with the benefits related to GitHub use in industry \cite{dabbish2012social}.

%better grading from instructors
Moreover, some students described the potential for better grading methods as a benefit of the transparency of activities on GitHub, despite these courses not utilizing the tool for grading. Compared to the traditional way of assignment submission where an assignment is handed in as a complete product when it is due, GitHub offers instructors the opportunity to keep up-to-date with assignments and projects, giving feedback while they are in progress.

\textbf{Beyond the Course} \\
%practice in tool
Supporting the findings from the instructor interviews in the previous chapter, most of the students interviewed described being exposed to GitHub and its features as a benefit to using the tool in a course. As such, the exposure to GitHub and `The GitHub Way' may result in some transferable skills for their careers. Moreover, the popularity of GitHub means that student GitHub accounts become a part of their online presence \cite{treude2012programming}, which may serve an important role with potential employers.

%outside help
With GitHub's popularity, many developers are putting their code on the platform, both publicly available or private. When a course is publicly visible, the `walled garden' that traditional LMS tend to suffer from \cite{mott2010envisioning} can be overcome. Student projects, for example, could involve people from another community, or outsiders can contribute to the course material in some manner.

\textbf{Tool Literacy} \\
%privacy issues
An important note from some of the limitations that the students and the instructors described is the importance of understanding and being proficient with the tool. As an example, in discussing what considerations need to be made to design an effective workflow, students would discuss the difficulty of conducting courses with set assignments rather than open-ended projects. This was due to the way in which GitHub repositories have to be private or public, making it difficult to handle assignment submission.

However, some experience with the tool or some investigation of GitHub's recommended practices for using their tool in education would surface the possibility of using private repositories for each assignment. An instructor could introduce new assignments or make clarifications in a student's private repository if they were simply added as a collaborator. As such, it is important to consider that some of the limitations described by the students and by the instructor may be from unfamiliarity with using the tool, especially in a context it originally was not meant to serve.

%still best for open-ended projects because of contribution benefits


%\subsubsection{Material Dissemination \& Interaction}
%GitHub can offer new ways to participate in the course and in the course material as well as opportunities to open up learning for the outside, thereby placing a greater emphasis on student agency in their learning. These new activities are supported by GitHub's collaborative features such as Pull Requests, Mentions, and the openness and transparency. Moreover, for students in fields as technical as Computer Science and Software Engineering, the use of a DCVS tool to support classroom activities provides an invaluable experience in a way of working that many groups in the field utilize. At the very least, the mere exposure of the system to students, even for trivial tasks like material dissemination, lets them learn about the tool and its features. This is reflected by many of the students' responses, from those who had very little prior experience to GitHub to those that believed that students should be given exposure to the tool in earlier courses.

%In examining the Contributing Student Pedagogy put forward by Hamer et al. [cite 2008], GitHub provides many of the characteristics of the pedagogy, where, for example, students are switching roles from passive to active and the focus on student contribution. %CSP tools 2011

%This is evidenced by student responses to the interviews, where students would claim more enjoyment or engagement thanks to the ability to see other groups' work or make changes or fixes to the course material. However, it is important to note that like any technology, meeting these characteristics requires the stakeholders to \'buy-in\' and use the relevant features of the tool to support this pedagogy. It's possible that there were different levels of enthusiasm for the tool between the two courses because of the difference between how it was utilized in labs, where the first case (SE) would require students to post, leading to looking at others' responses, while the other case (DS) would not, requiring only a demo to the lab instructor instead.

%\subsubsection{Open Participation}
%While traditional LMS can replicate the open discussion style that GitHub necessarily has, there are features in GitHub that lend itself well to student participation. In terms of the workflow in these courses where lab assignments involved open discussions as \'Issues\', users can \'Mention\' specific people to alert them, a feature not often seen in traditional discussion boards and forums. This has a few implications: it's a way to easily reach specific people publicly to comment on their work, it's a way for students to give credit to others, and it's an easy way to publicly communicate between student and professor. This lends itself well the participatory culture [cite Jenkins], where student contributions take an important part of the work. GitHub can therefore serve as a good tool for letting students be a part of each others' education in a social constructivist manner [cite Kim 2001].

%Significantly, while LMS may be too focused on the administrative side of education [cite Mott, Garcia-Penalvo, etc.], GitHub on its own may not offer enough administrative functions to be useful. This is what initially led the course instructor to use two different tools for these courses to begin with, and the students seemed to agree that the lack of features such as a Gradebook, Assignment Submission, and Announcements could hinder the use of GitHub as a learning tool. This means that future uses of GitHub may always be relegated to being a second tool, rather than fitting all the needs of a course. Of course, as a tool geared towards development rather than education, GitHub need not serve these purposes. However, this means that future tools that might address these benefits and these drawbacks of GitHub would need to ensure that the administrative functions are not ignored.

%Future tools would also need to address the privacy issues highlighted by many of the students. While to an extent, having the work publicized is a good thing, some students had reservations regarding activities like discussions, questioning why certain activities had to be public. %TO DO

%\subsubsection{Project Work}
%The third piece of the puzzle in these cases was the use of GitHub for group and individual projects. It should be stressed that using GitHub in this manner requires a certain type of course, one in which assignments and projects are more open-ended with no set answer. Otherwise, it becomes problematic to prevent students from getting each other's answers and copying from each other.

%The most important benefit GitHub provides to the students interviewed was the experience in working in such a manner. While material dissemination using GitHub could provide a good introduction to the tool amongst other benefits, many of the development-minded students gained good experience by using GitHub to develop their projects, learning about that way of working that is so prevalent in the industry. Beyond the experience, students would also gain a portfolio that can be easily shown off to prospective employers.

%Beyond that, GitHub provides a space that extends beyond the reach of the classroom. A student can utilize it so that outsiders can contribute to their work, which for some can be very rewarding. From the instructor's perspective, it's trivial to see exactly what work the student did and what was brought in from the outside, giving this potential use few downsides. It goes the other way as well, where students can easily reuse work from the outside as a part of their code or as inspiration to their code. This lends itself very well to the more open learning platforms discussed in the background section. %I will cite some stuff here..

%Finally, although it wasn't part of the workflow for these cases, students were excited about the potential of using version control as potentially a submission platform. This could have implications for better feedback from instructors, for convenience of being able to help out remotely, and for instructors to track student work.

\section{Limitations}
In this section, the limitations and the threats to validity of this study are outlined.

\subsection{Internal Threats to Validity}
Internal validity is concerned with biases within a study \cite{creswell2013research}. This is threatened when a researcher's actions and biases affect the work done in each process such as during data collection or during data analysis.

In this study, I was the sole researcher, and as such, my biases may have played a role in both data collection and in data analysis. The semi-structured nature of the interviews mean that I would often go off-script to probe further, and this may have resulted in leading questions.

Moreover, my recruitment methods listed earlier in the chapter may have biased the population: by searching for instructors teaching appropriate courses, I first approached instructors I knew to perform a case study, which may have introduced a bias as compared to finding a case that was already intending to use GitHub as a learning tool. This resulted in a less-than-optimal use of GitHub (as described by many of the students interviewed) because the instructor had little prior experience with GitHub as a tool. As well, because of this inexperience, I would give the instructor advice or resources on possibilities of how they can use GitHub to meet a goal - I couldn't, however, directly give them step-by-step directions without the study turning into action research.

In the data analysis, there was no inter-rater reliability because I was the sole researcher in this study. As such, biases may have been introduced in my selection of the themes. Having multiple raters analyze the data would have introduced more perspectives and interpretations, which would have reduced the potential biases in analysis. Unfortunately, this study suffers from a single-rater limitation.

Finally, the opportunistic nature of recruitment meant possible biases in multiple ways. With only one instructor teaching two courses, this study is limited from having no points of comparison. As well, opportunistic recruitment may have meant that the students willing to interview were students who felt strongly about GitHub in either direction - those who may have had insights but had no strong opinions may have chosen not to participate. Therefore, it cannot be assumed that the views of the interviewees represent the rest of the class.

\subsection{External Threats to Validity}
External validity is concerned with the extent to which the findings from this work can be generalized and to what extent the findings are of interest to people outside the case \cite{runeson2012case}. As a case study, it cannot be assumed that these cases can be generalized to the use of GitHub in Education. However, many of the findings are reflected in other studies that use similar tools for classes, such as Kelleher's study on Git and GitHub \cite{kelleher2014employing} and Haaranen and Lehtinen's study on Git and GitLab \cite{haaranen2015teaching}. %Another external threat to validity, is that because of the instructor's lack of experience with using GitHub, these findings may not be of interest to those who are familiar and can visualize a specific workflow for using GitHub in the classroom.
