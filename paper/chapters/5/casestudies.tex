\chapter{The Student Perspective}

After having gained insight on why and how educators use GitHub to augment their classes, as well as the benefits and drawbacks of using such a system from the instructor perspective. From the interviews conducted with various educators, the instructors explaiend multiple benefits such as the ability to better monitor student work and the ease in which they can reuse and remix course material from other instructors. However, they noted other benefits which have impact on their students, benefits such as the relevance of GitHub to the careers of the students and that using GitHub allowed them to encourage students to participate in changing the course material or discussing artifacts in the course like deadlines. Having heard these potential benefits to students, I decided to pursue that direction and see, firstly, how students feel about the current learning platforms and tools used in their classes, and then explore their perspective on GitHub as such a platform. We felt from that work that GitHub could, in the future, serve as

%check this paragraph
The result was to conduct a study in which GitHub in ways similar to Learning Management Systems, where the instructor would disseminate material and the students would have discussions on GitHub. Few studies exist that explore the student perspective of tools used in higher education and learning management systems beyond usability. In this instance, my questions were more exploratory as I wanted to explore how students found GitHub's effectiveness as a learning platform, in the process gathering their perspective on current learning tools and platforms used in the University and how GitHub fits in with those. It was important to place a focus on the students as learning tools become more geared towards student participation and contributions, as advocated by much of the research [cite a bunch of studies from Ch2].

\section{Research Questions}
The research questions addressed during this study include:
\bigskip
How do students feel that the use of systems like GitHub can benefit their education? We've seen evidence that the use of GitHub can be beneficial in a number of ways for educators. The natural progression was to see student perceptions on this.

\bigskip
What are the challenges for students related to the use of GitHub in their classes? When adopting a new tool, particularly one not originally tailored for education, there's bound to be some friction involved. We aimed to identify these so as to make recommendations towards designing a system more suitable for education.

\bigskip
How do students feel that GitHub as a tool must be improved to become more useful for education? As well, how do students feel the GitHub way of working can evolve to better suit education? As mentioned, GitHub and similar systems are not designed with education in mind

\bigskip
How do students feel the GitHub approach and GitHub as a tool in comparison to more traditional Learning Management Systems? Specifically, we aimed to gain insights on student perceptions on currently used educational tools such as Learning Management Systems like Coursespaces (Moodle) and Connex (Sakai) and GitHub's potential as such a portal for student interactions with the course.

\section{Research Design}
Consistent with the study conducted with the instructors, we approached this study with a qualitative approach. As Creswell [cite Creswell 2003] suggests, a qualitative, and therefore exploratory, approach best suits research when a concept or phenomenon requires more understanding because there's little pre-existing research. Moreover, continuing with a qualitative approach to be consistent with the previous study is important because the aim was to investigate similar outcomes from the same phenomenon, but from another perspective. This also continues the constructivist approach with which the research has been conducted.

Yin [cite Yin book] introduces case studies as \"an empirical inquiry that investigates a contemporary phenomenon within its real-life context, especially when the boundaries between phenomenon and context are not clearly evident\". Case study design, according to Yin, should be used when (a) the study seeks to answer \"how\" or \"why\" things happen; (b) the study is focused on the natural behavior of participants; (c) the context is important for the study; or (d) there are no clear descriptions of what is happening between \"the phenomenon and context.\" Most of these reasons apply to the nature of the research questions asked in this study, and as a result, the case study design was chosen for this work. Specifically, the study was exploratory, as it served as an early investigation on the phenomenon of GitHub in the classroom and to potentially build new theories or derive new hypotheses [cite Easterbrook et al.].

Specific to software engineering, Runeson [cite Runeson] defines case studies in software engineering as \"an empirical enquiry that draws on multiple sources of evidence to investigate one instance (or a small number of instances) of a contemporary software engineering phenomenon within its real-life context, especially when the boundary between phenomenon and context cannot be clearly specified.\" In this work, we aimed to draw on muliple sources of evidence - multiple students and instructors - to investigate the phenomenon of the GitHub Way approach to classes. As well, in this instance, it was difficult to separate this phenomenon from the context of educational tools in the classroom. It's important to learn student perspectives on the context as well, so as to explore the effectiveness of introducing a tool otherwise not built for the classroom.

\subsection{Recruitment}
This study involved the main stakeholders in a course as participants: the professor, lab instructor, and students. The learning tools used in the course directly involves and impacts these stakeholders, and as such, answering our research question must involve getting their perspective and opinions on said tools. I wanted to get their perspectives while the course was ongoing, so that they may recall recent experiences and provided their opinions on GitHub as a learning tool based off those experiences.


%or is this single case with embedded units?
%http://www.nova.edu/ssss/QR/QR13-4/baxter.pdf
For this study, we were opportunistic in finding cases, seeking instructors who could and would try using GitHub, a tool not often used in university classes, for the course. We recruited a professor who wanted to try the tool, and as she taught multiple courses, it set up the study for a multiple-case design. Having multiple cases, it would allow us to explore any possible differences between the two cases, with the goal being to replicate findings across cases [cite Yin]. The two course were Distributed Systems (DS), a Computer Science course that consisted of both undergraduate and graduate students, and Software Evolution (SE), a Software Engineering course that only consisted of undergraduate students. Both classes were similar in size (30-40 students) and in learning activities (two projects, labs every week).

When the term began, I attended one of the early lectures to describe the study and to recruit students for future participation. They would sign up with their names and emails with the knowledge that participation will be voluntary and that I would only contact them if they signed up to participate. As the course was wrapping up, I attended one more lecture to recruit some more participants in the same manner.

\subsection{Data Collection}
Data collection began with a preliminary survey to the students asking them to discuss their thoughts on learning tools in general and on GitHub as a learning tool. This was distributed to all students who signed up to participate, and received x respondents from the DS course and y respondents from the SE course.

As the main means of collecting data, I would interview the participants who were happy to discuss their experiences and opinions. Interviews began midway through the semester and would continue throughout. Most interviews with the students were one-to-one; however, due to scheduling reasons, some students requested to be interviewed as a group of 2 or 3. Interviews with the students lasted 20-30 minutes and were all conducted face-to-face in a meeting room. Audio from every interview was recorded with participant consent, and I would take notes on the participant's responses for reference. The interviews were semi-structured based on 10 guiding questions and I could dig deeper with additional questions as deemed appropriate. This supported the exploratory nature of the work, and allows for discovery of interesting insights.
%citation for semi-structured interviews; runeson?

The instructors were also interviewed. The course instructor was interviewed 3 times, once at the start, once in the middle of the semester, and once after the course concluded. This was useful in guaging her perspective throughout regarding the degree in which she found using GitHub useful and effective for her needs, as well as to see what she struggled with. As well, lab instructors (one for each course) were interviewed towards the end of the course, to find out firstly how they utilized GitHub in their labs and secondly to discern their opinions on its effectiveness towards the learning activities they engaged the students in. Interviews with the instructors had a similar format to those with the students: semi-structured, 20-30 minutes long, with around 10 guiding questions.

%follow-up survey

\subsection{Data Analysis}

\section{GitHub Use}
The professor opted to use GitHub in the same way for both courses, using it in three pivotal ways: material dissemination, lab work, and project hosting. Advanced uses from instructors we spoke to in Chapter 4, such as utilizing Pull Requests and Assignment Submissions, were not used for these courses. The instructor was aware of some of these features; however, as a novice user of Git and GitHub, she didn't feel confident using it beyond her knowledge.

The main use case was material dissemination: the professor hosted a repository which all students could access to find the work they had to do for any given week. The instructor would update this repository weekly, adding in lab assignments, links to readings, and the student homework for the week. All of this would be organized into a calendar table made from Markdown, and visible right on the home page of the course repository as a readme.
%screenshot

The other main use was in the \'Issues\' page of the repositories, where all labs (2-3 hour long sessions once a week separate from the course lectures) were hosted. These labs would often involve either using a tool and reporting results or student project work and giving feedback in some form to each other. A dedicated issue would be posted for each lab, similar to a forum post, and students would then make comments on these issues based on their lab work.
%Lab assignments, however, were somewhat different between the two courses, where for Course #2 (Distributed Systems), labs tended to have more complex instructions

The last use was for project hosting. Although GitHub use was not mandatory, most projects were hosted on GitHub and individual repositories. These repositories would be public, so others in the course (and outside, in some courses), can view the work and give feedback. %listed on Coursespaces

\section{Participants}
The two courses used as cases were courses named Software Evolution (C1) and Distributed Systems (C2). We were able to conduct inteviews with x students from C1, and with y students from C2. The main distinction between the two courses were that C1 was an undergraduate-only Software Engineering (SENG) course whereas C2 was a Computer Science (CSC) course with a mix of undergraduate and graduate students. Otherwise, as outlined above, the courses were laid out in a similar manner. Table z summarizes the students who participated in interviews.

%table here

\section{Qualitative Findings}

The research findings are presented in accordance to the research questions of the study. From the analysis, several common patterns in themes in the responses of the participants as they came up. Beyond the themes, some interviewees gave unique feedback that were not mentioned by others, but were noteworthy for several reasons - for example, they might speak to the potential of the tool or are unique drawbacks that only few people experienced. These will be highlighted appropriately and reasons will be given for why they are noteworthy answers.

%table of themes?

\subsection{RQ1: How do students feel that the use of systems like GitHub can benefit their education?}
- Introduction to the tool and the way of working (for personal use)
To preface the first few themes, we must reiterate that GitHub as a tool is, in the current landscape of software development, one of the most popular tools for working, boasting 9 million users and xyz million projects. As such, it becomes almost essential for developers to be familiar with at least this way of working, particularly when working on collaborative, multi-person projects.

Students came into the course with varying degrees of experience with GitHub, as shown on table x. Some hadn't used it at all, while others were very knowledgeable about the tool either through their own uses, through group projects for other classes, or through co-op jobs. Many, at least those who have done the majority of their undergraduate studies in the University of Victoria, had some familiarity with Subversion, though not necessarily Git, as Subversion was taught in a second year course.

Many of the interviewees mentioned that the use of the tool in class provided a good introduction to the tool for the students: \textit{``Yeah I think it's pretty good. I mean one thing is that because I'm using it in class, it's made me learn the tool, and so now that I've used it outside of class, and that's where the big takeaway is that I've been able to transfer those skills, I've done some other projects just on my own time using GH.''} [P3] Others felt that in general, it was a good way of learning that way of working, which is beneficial for their careers: \textit{``I think [learning about the tool will] be very useful. Especially if I wanna work with any, open source, any newer companies, a lot of them have a lot of open source projects. Just knowing about version control and stuff like that is gonna be... really useful in the future.''} [P6]

Those who had little experience with using Git and GitHub were given a good introduction to the features of the tool and how The GitHub Way works, particularly when they used it for their group projects. Others familiar with using Git and GitHub cited others in the class or in their groups that initially weren't familiar with it and acknowledged the way in which the class allowed them to experience this tool that many felt was important to know about as a software developer or as a computer scientist.

\textit{``well I think the first thing is not quite everybody had used Git before. So for some people it was a bit of an introduction to it, but I think that's definitely a good thing. Better now than going on a coop, and the first day being like, \'here, here\'s the git repo'.''} [P4]

- Portfolio
As an extension to this, many believed that GitHub use in the course benefited them twofold. First, for those who were yet to be acquainted with GitHub, this introduction allowed them to see its features and benefits, with many of these students asserting that they intended to use GitHub more after the course is over. As well, for every student who would use GitHub to host and manage their projects in the course, many enjoyed the fact that their completed projects would be on their GitHub account publicly viewable for anyone to see. When asked about the motivations behind putting their work on GitHub, P8 explains: \textit{``to actually show to people and maybe have some people collaborating or maybe have employers see that I have worked on stuff.''} [P8] Others describe the convenience factor: \textit{``I know that when you're trying to help somebody out, you can always just say \'Check out my GH\', I know I've done that with a few of my buddies... and I don't have to search through my files, it's just on GitHub, and you look on there. It's a good organization tool.''} [P10]

The reasoning many interviewees gave for this being a big benefit was that GitHub could serve as something akin to a portfolio for the students where their projects and their code hosted publicly could give them an advantage when job hunting. It has been acknowledged [cite] that many employers now look at GitHub accounts to help with hiring; in fact, some interviewees even had potential employers asking them to show their GitHub account: \textit{``...these days I see that employers also want to see your GitHub page. Like, while I was giving an interview for my coop, he did actually go into my GitHub profile and try to compile some of my code, so they do want you to have some online presence on GitHub. So it does help you in that, and since it's been used so widely, so, using it is necessary I think.''} [P5]

\textit{``It's always good for.. well I believe it's good for future employers. I remember I put directly on my resume saying you can check out the work I've done on GH. I included the link right on there and every person I handed my resume to were just like \'hey, fantastic!\'... it's a good way to get your skillset out there.''} [P10]

As such, that students were able to a) be introduced to GitHub as a place to store their code and b) use GitHub as a portfolio of sorts where they can show off their projects to future potential employers was, for many a big benefit of using the tool in this manner. All of the interviewees asserted that they would use GitHub outside the class, whether or not they've had experience with it beforehand. \textit{``I've actually used it more now that since I've started using it in class actually.''} [P3] \textit{``.''} [P11]

- Student participation (changing material, tagging others in discussion, etc.)
In the previous chapter, one of the benefits that instructors claimed that their use of GitHub provided them over traditional learning management systems was the ability to allow their students to make changes, fixes, or suggestions to the course material. Traditionally, this could be done by speaking to the instructor either in-person or by email, and the instructor could then make the changes as necessary. With GitHub, however, the students are able to make Pull Requests, changing the material themselves and asking the instructor to \'accept\' or \'close\' (reject) their Pull Request as they deem appropriate.

However, in the case of these two courses, throughout the semester, only three Pull Requests were made to make fixes to the material. What's more, they were all made in the first month of the course and by only one student who was well-versed in GitHub and in both courses. His explains his reasoning:

\textit{``I like being able to fix the mistakes that she might make like with a bad link or something by making a PR... I really like being able to do that because you know, it makes me feel a little more involved.''} [P2]

However, early complications made this student more hesitant to continue participating in this manner: %add quote.

Other students would acknowledge this is a nice benefit of using GitHub in this manner, claiming that it's something they could see themselves using had they known that it was an option. When asked about changes to make with the course and its workflow, P4 answers: \textit{``Like maybe if we can incorporate PRs in one way or another. Maybe with the idea I mentioned before, of having everyone contributing to the main repo... That would be neat and beneficial.''} [P4]

As such, some believed the professor should have given indication that this was an option. %quote
However, this would have been difficult given the instructor's relative inexperience with the tool. %change this sentence

Discussions were another way that students felt was a good way of participating in the course. Although discussions are a part of multiple other tools, GitHub's discussions, via the Issues feature, provided an advantage that others didn't - Mentions. By typing someone's username preceded by the \'@\' symbol, the user gets \'mentioned\', and this has two implications: a) other users reading the comment know who the comment was addressed to and b) notify the user being mentioned that someone mentioned them in a comment.

In the Issues for the classes (which acted as the Labs), many students would use it simply to sign their work or acknowledge their group members as part of their answers. However, during labs where interactions were more encouraged, such as when they were set to comment on each others' work, students would use this Mentions feature, which would notify those mentioned that a comment was directed at them. Students found this useful: %quote

- Openness
%group members held accountable
Another benefit provided by GitHub use came from its Transparency, and this served multiple purposes. The benefit cited most came from those who used GitHub to manage their group projects, where they could easily see if, and indeed when, their partners submitted work. Not only does their repository have an account of when each change was made, but users also see these changes being made on their News Feed as they log in. This helped them keep up with each other's work, and is a commonly cited benefit of using systems like GitHub [cite Dabbish et al.]:

%quote

Indeed, this also helped them keep themselves and each other accountable, so they know exactly how much work each member of their group put in: %quote

%able to see when repo is updated
Beyond that, having the course materials and schedule on the GitHub repository was helpful to some students thanks to the Transparency features of GitHub. If a change has been made or if the professor added something new to the repository, they would be informed in their News Feed and, if they are subscribed to the repository, by an email sent to them.

%quote

Unfortunately, some experienced issues with this, particularly when they were subscribed to the repository. This will be covered later when their responses pertaining to Research Question 2 are highlighted.

%seeing other groups work
As well, the open and public nature of GitHub meant that students could easily see each others' projects, allowing them to get inspiration from others, keep tabs on their progress as pertaining to other groups, and give and receive feedback to each other. Though most only looked at other group's projects when mandated by a lab assignment, some interviewees found this aspect of group work helpful: %quote

%seeing public repos for inspiration or for using their libraries

Others found this merely interesting, and explained their reasons for why it's not necessarily an important feature for them: %quote

- Distributed version control, work from any computer
Though not necessarily related to how GitHub was used in their class, many students described advantages to using GitHub to host their school work, including their projects for this class. That is, students enjoyed being able to do their work from any computer:

%quote

- Potential for outsiders to participate in their work (unique)
There was one benefit that, although was mentioned by only one participant, was noteworthy to consider as highlighting the potential of using GitHub for these purposes. One student was highly active in the community of a certain programming language (Rust). For their project, they were building something highly related to the language and therefore he advertised his work to the community, who would go on to help his work in multiple ways:

%quote AH

\subsection{RQ2: What are the challenges for students related to the use of GitHub in their classes?}
- Either completely public or completely private
While the public nature of GitHub provided advantages from some, others acknowledged the potential issues that such a way of working is sometimes not appropriate for a class environment. One person put it eloquently: %quote 'strength' is that it's completely public, 'weakness' is that it's completely public (BH?)

Most interviewees didn't mind that the class repository and their project work was completely public. In fact, while course work submission was split into two different tools, GitHub and CourseSpaces, most mentioned that they would have preferred it to be all on one tool, even if it was public in GitHub. However, students acknowledged multiple problems with the work being public.

%not everything is 100% effort
First, students mentioned that although they would ideally put 100\% effort into all their submissions, including lab assignments and comments, this is not always realistic due to the time crunch students face. %quote

Although one or two students felt that %quote if it's not your full effort, why hand it in?
, the majority that had an opinion acknowledged that sometimes, students rush through their work, and they therefore might not want work of that nature to be public. %quote.

%not everything is of interest to public
Beyond this, some felt that some of the work done on the course repository wouldn't even be of interest to the public or to potential employers, and as such they saw no need for the repository to be public. %quote

%first group's submissions rule all
Finally, many acknowledged that this way of working where everybody in the class can see submissions simply could not work in other types of courses, particularly those in which assignments have only one solution. %quote

In fact, some saw this effect even on the lab submissions in this course, even though lab work typically varies from group to group. %quote
But for some, this is advantageous. %quote

- Multiple tools is chaotic
Because of issues posted above, the instructor for these two courses decided to split the course work into two tools, GitHub and CourseSpaces. The idea was to put work that is more appropriate in a private space, such as responses to the course readings, out of the public eye. Unfortunately, most interviewees were not happy with this split. %quote. %expand

- Collaboration needs to be mandated
One issue multiple students described with the use of GitHub in their course was that it just was not utilized to its full potential. As such, some of the benefits they described to using such a system and to using that way of working in a classroom were only possibilities. A lot of the collaborative features needed to be mandated to have any effect. As for the features that weren't mandated such as the Pull Request features, many students felt that the features had potential, but needed to be mandated to see any real use.

%quotes

%explain effectiveness

%therefore, GitHub only equips instructors and students with the ability to do any of this, not guaranteed

- Not enough education on Git/GH
Another issue that many students described with using GitHub in the class is education - there were wildly varying degrees of experience with and knowledge of GitHub and it's features, which presented difficulties with its use. For one, some asserted that many of the benefits, such as the ability to make Pull Requests on the class material, were unknown to those less experienced with the tool, and as such was not taken advantage of. If the professor did not set a precedent for that behavior, it would not be used. %quote.

However, most of the students who were asked mentioned that the course could have benefited from more education on Git, GitHub and what they can do with it. Many said that they could have hosted a lecture or a lab dedicated to educating them about the tool, perhaps at the beginning of the course or as an extra. %quotes

Interestingly, students also asserted that there should be a bigger emphasis on version control systems such as GitHub in the undergraduate level at University of Victoria. Some believe that students should be getting an account pretty quickly after their first introduction courses to Computer Science. %quote
They believed that students should be putting their code into GitHub early on as it provides numerous benefits for them. %quote EB

%However, those who weren't very experienced with the tool to begin with asserted that they would continue using it beyond the class and into their personal and group work, stating that the class served as a good introduction to the tool

- LMS Features (Notifications)
Several students described notifications as a weak point of GitHub. As it stands, the only way to get notifications from a course repository is to \'Watch\' the repository. \'Watching\' provides two different options: to get a notification (and an email) when the user is mentioned in issues or commits, or to get a notification (and an email) when anything at all happens in the main branch (master), something happens on the comment, or when someone makes or accepts a Pull Request. Unfortunately, this presented problems to some students.
%check that

%watching all means too many notifications

%mentions, not enough

%most studnets asked said they had to just visit the repo, some didn't know about watching

%this RQ should be changed
%How can GitHub go beyond what traditional LMS offer for the student benefit?
\subsection{RQ3: How do students feel that GitHub as a tool must be improved to become more useful for education?}

\subsection{RQ4: How do students feel the GitHub approach and GitHub as a tool in comparison to more traditional Learning Management Systems?}
One of the goals of this research study was to see how GitHub would fare as a tool that served a purpose similar to those of learning management systems (LMS), particularly from the student perspective. Instructors that we spoke to in Chapter 4 described using GitHub in such a manner, which made this use case worthy of investigation. The instructor for these case studies believed in the benefits of this way of working as well. As highlighted in Chapter 2, the primary purposes of traditional LMS include material dissemination, announcements, assignment submission, and discussion among the students and instructors.

- Discussions
The feature that came up the most in interviews when GitHub was being compared to traditional LMS at UVic (Connex/Sakai, Coursespaces/Moodle) were discussions and forums. For both courses, the students had two main places where they could discuss and comment - on the course repository's \'Issues\' area and on Coursespaces' Forums. The professor separated these, by assigning all lab work and discussion to GitHub and the assignment readings and comments to Coursespaces.

Students were generally receptive to the use of issues as posts, particularly as it offered some flexibility otherwise not seen on regular forums. For one, students liked the ability to \'Mention\' others and be notified when they themselves are \'Mentioned\'. %quote

As well, \'Issues\' offer some flexibility in that when student projects are hosted on GitHub, students are able to help each other out in each others repositories by filing issues. %quote

- Assignment Submission
Another way in which GitHub may compare to traditional LMS was in the way assignment submission can be handled. In traditional LMS, assignment submission tend to be a private upload of relevant files to the instructor. This simplifies the process, ensuring that there is only one submission when the assignment is new.

These courses did not utilize GitHub for any form of submission of their code. However, when asked regarding GitHub serving as an LMS, many students highlighted the lack of formal submission features in GitHub. %quote

As highlighted earlier, because of the transparent nature of GitHub, assignment submission may be difficult when students are working towards the same solution. %quote
However, some students attempted to brainstorm workarounds. %quote

Nevertheless, it seemed that this was one area where GitHub may be lacking as a potential educational tool.

- Simplicity of interface

- Offers more opportunities for engagement

\section{Discussion}
