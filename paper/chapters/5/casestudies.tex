\chapter{The Student Perspective}

After having gained insight on why and how educators use GitHub to augment their classes, as well as the benefits and drawbacks of using such a system from the instructor perspective. From the interviews conducted with various educators, the instructors explaiend multiple benefits such as the ability to better monitor student work and the ease in which they can reuse and remix course material from other instructors. However, they noted other benefits which have impact on their students, benefits such as the relevance of GitHub to the careers of the students and that using GitHub allowed them to encourage students to participate in changing the course material or discussing artifacts in the course like deadlines. Having heard these potential benefits to students, I decided to pursue that direction and see, firstly, how students feel about the current learning platforms and tools used in their classes, and then explore their perspective on GitHub as such a platform. We felt from that work that GitHub could, in the future, serve as

%check this paragraph
The result was to conduct a study in which GitHub in ways similar to Learning Management Systems, where the instructor would disseminate material and the students would have discussions on GitHub. Few studies exist that explore the student perspective of tools used in higher education and learning management systems beyond usability. In this instance, my questions were more exploratory as I wanted to explore how students found GitHub's effectiveness as a learning platform, in the process gathering their perspective on current learning tools and platforms used in the University and how GitHub fits in with those. It was important to place a focus on the students as learning tools become more geared towards student participation and contributions, as advocated by much of the research [cite a bunch of studies from Ch2].

The research questions addressed during this study include:
- How effective is GitHub in supporting various learning acitivities and processes? %nope. change this.
- What are the benefits and limitations of using GitHub as a learning tool for students?
- What are student perspectives on the effectiveness of GitHub as a learning platform, especially in comparison to traditional learning management systems?

\section{Research Design}
Consistent with the study conducted with the instructors, we approached this study with a qualitative approach. As Creswell [cite Creswell 2003] suggests, a qualitative, and therefore exploratory, approach best suits research when a concept or phenomenon requires more understanding because there's little pre-existing research. Moreover, continuing with a qualitative approach to be consistent with the previous study is important because the aim was to investigate similar outcomes from the same phenomenon, but from another perspective. This also continues the constructivist approach with which the research has been conducted.

Yin [cite Yin book] introduces case studies as \"an empirical inquiry that investigates a contemporary phenomenon within its real-life context, especially when the boundaries between phenomenon and context are not clearly evident\". Case study design, according to Yin, should be used when (a) the study seeks to answer \"how\" or \"why\" things happen; (b) the study is focused on the natural behavior of participants; (c) the context is important for the study; or (d) there are no clear descriptions of what is happening between \"the phenomenon and context.\" Most of these reasons apply to the nature of the research questions asked in this study, and as a result, the case study design was chosen for this work. Specifically, the study was exploratory, as it served as an early investigation on the phenomenon of GitHub in the classroom and to potentially build new theories or derive new hypotheses [cite Easterbrook et al.].

%or is this single case with embedded units?
%http://www.nova.edu/ssss/QR/QR13-4/baxter.pdf
For this study, we were opportunistic in finding cases, seeking instructors who could and would try using GitHub, a tool not often used in university classes, for the course. We recruited a professor who wanted to try it, and as she taught multiple courses, it set up the study for a multiple-case design. Having multiple cases, it would allow us to explore any possible differences between the two cases, with the goal being to replicate findings across cases [cite Yin].

%stuff from Runeson

\section{Approach}
