\chapter{The Student Perspective}

After having gained insight on why and how educators use GitHub to augment their classes, as well as the benefits and drawbacks of using such a system from the instructor perspective. From the interviews conducted with various educators, the instructors explaiend multiple benefits such as the ability to better monitor student work and the ease in which they can reuse and remix course material from other instructors. However, they noted other benefits which have impact on their students, benefits such as the relevance of GitHub to the careers of the students and that using GitHub allowed them to encourage students to participate in changing the course material or discussing artifacts in the course like deadlines. Having heard these potential benefits to students, I decided to pursue that direction and see, firstly, how students feel about the current learning platforms and tools used in their classes, and then explore their perspective on GitHub as such a platform. We felt from that work that GitHub could, in the future, serve as

%check this paragraph
The result was to conduct a study in which GitHub in ways similar to Learning Management Systems, where the instructor would disseminate material and the students would have discussions on GitHub. Few studies exist that explore the student perspective of tools used in higher education and learning management systems beyond usability. In this instance, my questions were more exploratory as I wanted to explore how students found GitHub's effectiveness as a learning platform, in the process gathering their perspective on current learning tools and platforms used in the University and how GitHub fits in with those. It was important to place a focus on the students as learning tools become more geared towards student participation and contributions, as advocated by much of the research [cite a bunch of studies from Ch2].

\section{Research Questions}
The research questions addressed during this study include:
\bigskip
How do students feel that the use of systems like GitHub can benefit their education? We've seen evidence that the use of GitHub can be beneficial in a number of ways for educators. The natural progression was to see student perceptions on this.

\bigskip
What are the challenges for students related to the use of GitHub in their classes? When adopting a new tool, particularly one not originally tailored for education, there's bound to be some friction involved. We aimed to identify these so as to make recommendations towards designing a system more suitable for education.

\bigskip
How do students feel that GitHub as a tool must be improved to become more useful for education? As well, how do students feel the GitHub way of working can evolve to better suit education? As mentioned, GitHub and similar systems are not designed with education in mind

\bigskip
How do students feel the GitHub approach and GitHub as a tool in comparison to more traditional Learning Management Systems? Specifically, we aimed to gain insights on student perceptions on currently used educational tools such as Learning Management Systems like Coursespaces (Moodle) and Connex (Sakai) and GitHub's potential as such a portal for student interactions with the course.

\section{Research Design}
Consistent with the study conducted with the instructors, we approached this study with a qualitative approach. As Creswell [cite Creswell 2003] suggests, a qualitative, and therefore exploratory, approach best suits research when a concept or phenomenon requires more understanding because there's little pre-existing research. Moreover, continuing with a qualitative approach to be consistent with the previous study is important because the aim was to investigate similar outcomes from the same phenomenon, but from another perspective. This also continues the constructivist approach with which the research has been conducted.

Yin [cite Yin book] introduces case studies as \"an empirical inquiry that investigates a contemporary phenomenon within its real-life context, especially when the boundaries between phenomenon and context are not clearly evident\". Case study design, according to Yin, should be used when (a) the study seeks to answer \"how\" or \"why\" things happen; (b) the study is focused on the natural behavior of participants; (c) the context is important for the study; or (d) there are no clear descriptions of what is happening between \"the phenomenon and context.\" Most of these reasons apply to the nature of the research questions asked in this study, and as a result, the case study design was chosen for this work. Specifically, the study was exploratory, as it served as an early investigation on the phenomenon of GitHub in the classroom and to potentially build new theories or derive new hypotheses [cite Easterbrook et al.].

Specific to software engineering, Runeson [cite Runeson] defines case studies in software engineering as \"an empirical enquiry that draws on multiple sources of evidence to investigate one instance (or a small number of instances) of a contemporary software engineering phenomenon within its real-life context, especially when the boundary between phenomenon and context cannot be clearly specified.\" In this work, we aimed to draw on muliple sources of evidence - multiple students and instructors - to investigate the phenomenon of the GitHub Way approach to classes. As well, in this instance, it was difficult to separate this phenomenon from the context of educational tools in the classroom. It's important to learn student perspectives on the context as well, so as to explore the effectiveness of introducing a tool otherwise not built for the classroom.

\subsection{Participants}
This study involved the main stakeholders in a course as participants: the professor, lab instructor, and students. The learning tools used in the course directly involves and impacts these stakeholders, and as such, answering our research question must involve getting their perspective and opinions on said tools. I wanted to get their perspectives while the course was ongoing, so that they may recall recent experiences and provided their opinions on GitHub as a learning tool based off those experiences.

%discuss year, SE vs. CS, GH experience

\subsection{Recruitment}
%or is this single case with embedded units?
%http://www.nova.edu/ssss/QR/QR13-4/baxter.pdf
For this study, we were opportunistic in finding cases, seeking instructors who could and would try using GitHub, a tool not often used in university classes, for the course. We recruited a professor who wanted to try the tool, and as she taught multiple courses, it set up the study for a multiple-case design. Having multiple cases, it would allow us to explore any possible differences between the two cases, with the goal being to replicate findings across cases [cite Yin]. The two course were Distributed Systems (DS), a Computer Science course that consisted of both undergraduate and graduate students, and Software Evolution (SE), a Software Engineering course that only consisted of undergraduate students. Both classes were similar in size (30-40 students) and in learning activities (two projects, labs every week).

When the term began, I attended one of the early lectures to describe the study and to recruit students for future participation. They would sign up with their names and emails with the knowledge that participation will be voluntary and that I would only contact them if they signed up to participate. As the course was wrapping up, I attended one more lecture to recruit some more participants in the same manner.

\subsection{Data Collection}
Data collection began with a preliminary survey to the students asking them to discuss their thoughts on learning tools in general and on GitHub as a learning tool. This was distributed to all students who signed up to participate, and received x respondents from the DS course and y respondents from the SE course.

As the main means of collecting data, I would interview the participants who were happy to discuss their experiences and opinions. Interviews began midway through the semester and would continue throughout. Most interviews with the students were one-to-one; however, due to scheduling reasons, some students requested to be interviewed as a group of 2 or 3. Interviews with the students lasted 20-30 minutes and were all conducted face-to-face in a meeting room. Audio from every interview was recorded with participant consent, and I would take notes on the participant's responses for reference. The interviews were semi-structured based on 10 guiding questions and I could dig deeper with additional questions as deemed appropriate. This supported the exploratory nature of the work, and allows for discovery of interesting insights.
%citation for semi-structured interviews; runeson?

The instructors were also interviewed. The course instructor was interviewed 3 times, once at the start, once in the middle of the semester, and once after the course concluded. This was useful in guaging her perspective throughout regarding the degree in which she found using GitHub useful and effective for her needs, as well as to see what she struggled with. As well, lab instructors (one for each course) were interviewed towards the end of the course, to find out firstly how they utilized GitHub in their labs and secondly to discern their opinions on its effectiveness towards the learning activities they engaged the students in. Interviews with the instructors had a similar format to those with the students: semi-structured, 20-30 minutes long, with around 10 guiding questions.

%follow-up survey

\subsection{Data Analysis}

\section{GitHub Use}
%how was GitHub used in the class?

\section{Qualitative Findings}

The research findings are presented in accordance to the research questions of the study. From the analysis, several common patterns in themes in the responses of the participants as they came up. Beyond the themes, some interviewees gave unique feedback that were not mentioned by others, but were noteworthy for several reasons - for example, they might speak to the potential of the tool or are unique drawbacks that only few people experienced. These will be highlighted appropriately and reasons will be given for why they are noteworthy answers.

%table of themes?

\subsection{RQ1: How do students feel that the use of systems like GitHub can benefit their education?}
- Introduction to the tool and the way of working (for personal use)
To preface the first few themes, we must reiterate that GitHub as a tool is, in the current landscape of software development, one of the most popular tools for working, boasting 9 million users and xyz million projects. As such, it becomes almost essential for developers to be familiar with at least this way of working, particularly when working on collaborative, multi-person projects.

Students came into the course with varying degrees of experience with GitHub. Some hadn't used it at all, while others were quite familiar with the tool either through their own uses, through group projects for other classes, or through co-op jobs. Many, at least those who have done the majority of their undergraduate studies in the University of Victoria, had some familiarity with Subversion, though not necessarily Git, as Subversion was taught in a second year course.

Many of the interviewees mentioned that the use of the tool in class provided a good introduction to the tool for the students. Those who had little experience with using Git and GitHub were given a good introduction to the features of the tool and how The GitHub Way works, particularly when they used it for their group projects. Those familiar with using Git and GitHub cited others in the class or in their groups that initially weren't familiar with it and acknowledged the way in which the class allowed them to experience this tool that many felt was important to know about as a software developer or as a computer scientist.

- Portfolio
As an extension to this, many believed that GitHub use in the course benefited them twofold. First, for those who were yet to be acquainted with GitHub, this introduction allowed them to see its features and benefits, with many of these students asserting that they intended to use GitHub more after the course is over. As well, for every student who would use GitHub to host and manage their projects in the course, many enjoyed the fact that their completed projects would be on their GitHub account publicly viewable for anyone to see.

The reasoning many interviewees gave for this being a big benefit was that GitHub could serve as something akin to a portfolio for the students where their projects and their code hosted publicly could give them an advantage when job hunting. It has been acknowledged [cite] that many employers now look at GitHub accounts to help with hiring; in fact, some interviewees even had potential employers asking them to show their GitHub account. As such, that students were able to a) be introduced to GitHub as a place to store their code and b) use GitHub as a portfolio of sorts where they can show off their projects to future potential employers was, for many a big benefit of using the tool in this manner. All of the interviewees asserted that they would use GitHub outside the class, whether or not they've had experience with it beforehand.

- Student engagement (changing material, tagging others in discussion, etc.)

- Openness - seeing each others' work

- Distributed version control, work from any computer

- Potential for outsiders to participate in their work (unique)

\subsection{RQ2: What are the challenges for students related to the use of GitHub in their classes?}
- Either completely public or completely private

- Multiple tools is chaotic

- Collaboration needs to be mandated

- Not enough education on Git/GH

%this RQ should be changed
%How can GitHub go beyond what traditional LMS offer for the student benefit?
\subsection{RQ3: How do students feel that GitHub as a tool must be improved to become more useful for education?}

\subsection{RQ4: How do students feel the GitHub approach and GitHub as a tool in comparison to more traditional Learning Management Systems?}
- Discussions

- Simplicity of interface

- Offers more opportunities for engagement

\section{Discussion}
