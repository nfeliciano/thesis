\chapter{Methodology}

To investigate learning tools in computer science and software engineering education, a number of qualitative approaches were used. In using research methods that were largely exploratory, we would determine how using systems like GitHub can influence the educational approach for both educators and students.

..Discuss focus groups? G4E study?

As such, we employed a case study methodology, where we set to gain insights on student perception of current learning tools they use in the university computer science and software engineering courses as well as on using GitHub as a course portal. We sought out educators who would use GitHub in the courses they taught in any capacity in order to better understand student perceptions on the tool's place in an educational context as they experience it.

..Discuss case studies, Runeson's book, Yin's book

The professor we spoke to would use GitHub to teach two of her courses, providing two different cases for investigation. Our data collection involved two main techniques: interviews (with students and instructors) and surveys (with students). This section describes our research questions, study design, participants, data collection methods, and our approach to analyzing the data.

\subsection{Research Questions}
We strive to answer the following research questions:

\bigskip
How do students feel that the use of systems like GitHub can benefit their education? We've seen evidence that the use of GitHub can be beneficial in a number of ways for educators. The natural progression was to see student perceptions on this.

\bigskip
What are the drawbacks for students related to the use of GitHub in their classes? When adopting a new tool, particularly one not originally tailored for education, there's bound to be some friction involved. We aimed to identify these so as to make recommendations towards designing a system more suitable for education.

\bigskip
How do students feel about systems like GitHub in relation to other educational tools? Specifically, we aimed to gain insights on student perceptions on currently used educational tools such as Learning Management Systems like Coursespaces (Moodle) and Connex (Sakai) and GitHub's potential as such a portal for student interactions with the course.

\subsection{Study Design}
