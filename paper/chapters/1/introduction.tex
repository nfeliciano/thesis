\chapter{Introduction}

%- Introduction ramblings
%- Thesis point
%- Problem Statement
%- Thesis motivation
%- RQs
%- Framework?
%- Methodology
%- Thesis Organization

%GITHUB WAY GITHUB WAY GITHUB WAY

As technology continues to play a vital role in the education of post-secondary students, web-based tools have evolved in ways that make them useful for educational purposes. Instructors in classrooms both local and remote utilize a number of these technologies to disseminate material and engage the students, whether they are tools focused on education such as Learning Management Systems (LMS) or tools created for other purposes re-fitted for educational use such as various social media platforms. The term ‘e-learning’ encapsulates this, and from an educational standpoint, can be defined as “electronically mediated asynchronous and synchronous communication for the purpose of constructing and confirming knowledge” [cite Garrison book].

As a result of the Web 2.0 movement [cite O'Reilly], learning has placed a larger emphasis on social interaction through the use of social media or computer-mediated communication tools. Such tools have characteristics which afford users - in the case of education, students - more freedom to create, publish, and co-create content rather than just read. These have introduced affordances for students to be more than passive observers who absorb content as given to them; instead, they are able to participate and collaborate with others in gaining knowledge. As a result, learning can become more social and more engaging for the student.

Computer Science Education (CSE) and Software Engineering Education & Training (SEET) utilizes these e-learning tools in similar ways to other disciplines. However, tools and technologies play an important role in computing careers, particularly in software development, for reasons such as the importance placed on collaborative processes using such tools. Therefore, it is crucial that e-learning tools evolve as to better accommodate the learning needs of students in this field, so as to better train and familiarize students in these processes specific to computer science and software development.

%discuss GitHub and DCVS in a paragraph here

This thesis looks at tools in CSE and SEET, specifically identifying GitHub's use in this context as a way to fill the gaps that exist in current tools used in education. For both educators and students, blah blah blah %beneficial for both sets of users, not just GitHub

%describe studies in a paragraph

\section{Problem Statement}
An important goal of computer science and software engineering education is to prepare students well for their future careers in industry. As many software projects today are increasingly collaborative, such as those in globally distributed projects or open-source projects, it becomes necessary for educators to prepare their students for a career in this environment. Yet, the concept of group and collaboration skills has been discussed by many researchers and educators, with assertions that students are generally not well-prepared with group work skills [cite Waite et al.].

While many are making adjustments to their curricula to account for this [cite Jayazeri, cite Coleman & Lang], another area worth investigation is the use of tools in computer science and software engineering education. Extending to other disciplines, web-based learning tools sees regular use in education as the main portal for students to engage with the course and its participants. Therefore these tools, which are traditionally Learning Management Systems (LMS), has the potential to serve an important point of interaction between not just the students and educators, but between the students themselves [cite Minds on Fire?]. Yet many of these tools are poorly equipped to handle collaborative tasks and student participation, with their focus on administrative and instructor tasks rather than student learning [cite McLoughlin?]. As such, a gap exists in learning tools, where their collaborative and participatory functions must be addressed, particularly in areas such as CSE and SEET where participation is an important part of the career landscape for students.

This thesis includes two studies which address this gap, where I look at the use of GitHub as an educational tool that can provide benefits for instructors in the way of remixing course content and for engaging their students, and benefits for students in the way of active participation in the course and as a way of learning a way of working that the industry uses. %MORE!!

\section{Motivation?}

\section{Research Questions}
This thesis aims to explore tools in computer science and software engineering education, particularly the use of a tool such as GitHub and its effectiveness in that context. The research questions involved are therefore exploratory, and they are:
1. What is the current landscape of tools used for Computer Science & Software Engineering education and what needs are not being met?
2. Is the GitHub Way a viable approach to education of CS/SE students and how does it compare to traditional learning tools?
2a. What are the benefits and weaknesses to that approach from an instructor's perspective?
2b. What are the benefits and weaknesses to that approach from the student's perspective?

\section{Framework}

\section{Research Approach}
%exploratory - justify!
I approached my research questions with a constructivist approach, where I would seek to construct knowledge and theories, relying mainly on participants' views [cite Easterbrook et al.]. A constructivist, according to Easterbrook, does not focus on verifying theories, but instead seeks to understand how people make sense of the world, and might build theories based on the context being studied. In this case, my goal was to seek student and instructor perspectives on educational tools, and particularly on a tool like GitHub in an educational context. Based on their perspectives, theories might then be formed regarding why a tool like GitHub might be beneficial for these students' educations.

My first research question was addressed through a thorough literature search, where I look at work regarding computer science and software engineering education, tools used in education in general, and tool use in cs/se education. I also look at literature surrounding GitHub and similar Distributed Version Control Systems (DCVS), particularly to explore their collaborative features and seek other instances where they have been used in educational contexts. In my search, I focused on educational tools, exploring what researchers believe typical LMSes lack and how they think those can be addressed.

The second research question is then addressed in two different studies - one involving instructors as participants and another involving courses as case studies. In both studies, the research methods used were congruent with exploratory research, as we sought to explore the perspectives of instructors and students on the use of GitHub as a platform for education. In this thesis, both studies will be described in separate sections, and the research methodology will be explained in more detail within.

\section{Thesis Organization}
%literature review

%study 1

%study 2

%results

%discussion

%As such, I aim to distinguish the gaps in such tools and to explain why GitHub and similar tools can provide means for students to be more involved and engaged in course content.

%..Done in the software development world, which usually serves as early adopters, particularly in education (using wikis, twitter, other social media as example)

%..DCVS and GitHub, use in Software Development world

%..CS/SE tools in education, what can be addressed
