\chapter{Introduction}

%- Introduction ramblings
%- Thesis point
%- Problem Statement
%- Thesis motivation
%- RQs
%- Framework?
%- Methodology
%- Thesis Organization

%GITHUB WAY GITHUB WAY GITHUB WAY

As technology continues to play a vital role in the education of post-secondary students, Web-based tools have evolved to cater to educational needs. Instructors in both local and remote classrooms utilize a number of these technologies to disseminate material and engage their students, including tools such as Learning Management Systems (LMS) that are focused on education, or tools created for other purposes used in an educational context, such as social media platforms. These technologies can be defined as tools for `e-learning', a term that can be defined as \textit{``electronically mediated asynchronous and synchronous communication for the purpose of constructing and confirming knowledge''} \cite{garrison2011learning}.

Coinciding with the Web 2.0 movement \cite{O'Reilly-What-2005}, education began to emphasize social interaction through the use of social media or computer-mediated communication tools. Such tools have characteristics which afford users more freedom to create and publish content. This allows for an approach to learning characterized by a \emph{demand-pull} model rather than a \emph{supply-push} model, and focuses on participation and providing students access to rich learning communities \cite{seely2008open}. These tools have introduced ways for students to be more than just passive observers who absorb content as given to them. Instead, they are able to participate in communities, and as a result, their learning can become more social and engaging. However, educators continue to use the traditional Learning Management Systems in place of or in conjunction with these Web 2.0 tools, where the opportunities for students to make contributions beyond discussions are minimal, as there remains a larger focus on the administrative functions rather than on student-centered actions.

%However, tools and technologies play an important role in computing careers, particularly in software development where collaborative processes and work rely on the use of various technologies. Therefore, it is vital in this discipline that e-learning tools evolve to accommodate the students' learning needs, training and familiarizing them with these collaborative processes so prevalent in the field.
%needs citation
%http://www.researchgate.net/profile/Timothy_Lethbridge/publication/4250887_Improving_software_practice_through_education_Challenges_and_future_trends/links/09e41507da209abb93000000.pdf

Computer science and software engineering education utilizes these e-learning tools in similar ways to other disciplines. However, work in these fields often involve a large amount of collaboration with others, such as being able to examine, understand, and build upon a collaborator's work. The tools used for educating students in those fields will then need to support such activities, providing students training and practice in the skills related to collaboration such as communication and teamwork skills.

In the software development field, GitHub is a social code sharing service and version control system. It is an essential tool for many groups and projects that require collaboration, and has even seen utilization in areas outside software development, such as technical writing. The advantages of GitHub and similar tools include their awareness and transparency features, where collaborators can easily stay informed of others' work \cite{dabbish2012social}. As well, collaborators in a GitHub repository can be involved in a project in a number of ways, such as contributing to discussions regarding bugs and features, or making changes to a project itself and allowing other collaborators to review and accept their changes. This way of working is called `The GitHub Way' \footnote{\url{http://www.wired.com/2013/09/github-for-anything/}} as these features are not necessarily exclusive to GitHub and can be found in other Distributed Version Control Systems (DVCS) such as BitBucket. I expand on defining the GitHub way in Chapter 3.

This thesis explores tools in computer science and software engineering education, specifically identifying GitHub's use in this context as a way to address the weaknesses that exist in current e-learning tools. For both educators and students, working using The GitHub Way has both potential benefits and drawbacks, which this work aims to explore and identify---we try to determine whether or not `The GitHub Way' is a viable approach in computer science and software engineering education.

\section{Problem Statement}
An important goal of computer science and software engineering education is to prepare students for their future careers in industry. As many software projects today are increasingly collaborative, such as those in globally distributed or open-source projects, it becomes necessary for educators to prepare their students for a career in this environment. Yet, many researchers and educators assert that students are generally lacking in their group work skills \cite{waite2004student}.

While some are adjusting their curricula to account for these weaknesses \cite{jazayeri2004education} and giving students opportunities to develop skills in environments closer to real-world experiences \cite{coleman2012collaboration}, another area to investigate is the use of tools in computer science and software engineering education. Extending to other disciplines, Web-based e-learning tools are regularly used in education as the main portal for students to engage with a course and its participants. Therefore, educational tools such as Learning Management Systems (LMS) becomes an important point of interaction between not just the students and educators, but between the students themselves. Yet many of these tools are poorly equipped to handle collaborative tasks and student participation because of their focus on administrative and instructor tasks rather than on student learning \cite{mcloughlin2007social}. As such, a gap exists in traditional learning tools, where they are lacking in support for activities that allow students to participate and contribute to the learning experience. This is particularly important in areas such as computer science and software engineering education where participation and collaboration is an important part of the career landscape for students.

%This thesis includes two studies which address this gap, where I examine the use of GitHub as an educational tool that can provide benefits for instructors such as the ability to remix course content and various ways to engage their students, and benefits for students such as the ability to more actively participate in a course and to learn a way of working prevalent in the field.

This work studies instructors and students that attempted to use GitHub as an e-learning tool. While the instructors and students reaped many benefits from using GitHub in courses, there were drawbacks and challenges with using a tool not built for education. Conclusions from this work can help shape the development of future educationally-focused tools which, similar to GitHub, gives students the opportunity to contribute to the learning experience in multiple ways.

%here or in ch4?
\section{Motivation}
This project began when our research group noted the popularity of GitHub and how it was seeing widespread adoption in other areas \footnote{\url{\url{http://www.wired.com/2013/09/github-for-anything/}}}, transforming how people collaborate over a shared repository \cite{begel2013social}. One of GitHub's main strengths is in the awareness and transparency features it provides to team, project and community members \cite{dabbish2012social}. These features positively influence how people contribute to projects \cite{Tsay:2012:SMS:2141512.2141583}.

% GitHub for education Greg Wilson
The use of GitHub in an educational context is certainly not novel. A few years after GitHub's release, Greg Wilson, a well-known computer science educator, suggested that GitHub could be used for learning materials despite its limitations, citing remixing as the primary benefit of using GitHub in this context\footnote{\url{http://software-carpentry.org/blog/2011/12/fork-merge-and-share.html}}.

\begin{quote}\textit{Would it be possible to create a ``GitHub for education?'' Right now, I think the answer is ``no'', because today's learning content formats make merging hard. Whatever a ``GitHub for education'' would look like, it would not be yet another repository of open learning materials. There are lots of those already, but almost all their content is write-once-and-upload, \ldots rather than sharing course content in a reusable, remixable way.}\end{quote}

In 2012, he elaborated on this idea\footnote{\url{http://software-carpentry.org/blog/2012/04/github-for-education.html}}:

\begin{quote}\textit{``GitHub for Education'' isn’t necessarily, ``Let’s put educational materials in GitHub'', but rather, ``Let’s facilitate a culture of spontaneous-but-structured collaboration and improvement.''}\end{quote}

Wilson recognized the limitations of the majority of LMSs in the difficulty for instructors trying to reuse and share materials. This is the type of problem that the software development community had already solved using tools such as GitHub. This brings into question whether such tools originally built for software developers have their place in other fields that have similar needs such as education.

In an effort to promote GitHub to higher education, GitHub launched the GitHub Education Website---\url{https://education.github.com/}---in 2014. This promotion offers support to students and instructors using GitHub for educational purposes, and signals GitHub's intention to become a more useful educational tool.

% Software Developers did a bunch of things first
%look up more stuff here
Consequently, software developers are sometimes described as the ``prototype of future knowledge workers'', as they are often the first to adopt new tools and techniques\footnote{\url{http://allankelly.blogspot.ca/2014/04/the-prototype-of-future-knowledge.html}}. Because software developers and programmers are able to create and change products to meet their needs, they more often than not become the early adopters of tools that other fields would eventually use. This is evident in examples such as e-mail, the Web, and wikis: tools that were invented and used by programmers, which later became significant tools in other fields.

%http://accu.org/index.php/journals/1474

% Always on learning
Moreover, programmers and software developers experience `always-on' learning, where they constantly have access to learning materials and content relevant in their field. In software development, knowledge is constantly evolving and developers often have to stay current with languages, libraries, frameworks, and tools. Unfortunately, courses are often self-contained, with little input or output to and from outside software development communities. This could potentially be addressed using tools in which users can interface with other communities and groups beyond the class.

% My experiences as a student?
%As a student in the University of Victoria, I experienced a lot of frustration .

\section{Research Questions}
This thesis aims to explore tools in computer science and software engineering education, particularly the use of a tool such as GitHub and its effectiveness in the educational context. These research questions are exploratory in nature: \\
\begin{enumerate}
\item What is the \emph{current landscape} of tools used for computer science \& software engineering education and what needs are not being met?
\item Is the \emph{`GitHub Way'} a viable approach to the education of computer science and software engineering students and how does it \emph{compare} to traditional learning tools?
\begin{enumerate}
    \item What are the benefits and weaknesses to this approach from an \emph{instructor's perspective}?
    \item What are the benefits and weaknesses to this approach from the \emph{student's perspective}?
\end{enumerate}
\end{enumerate}

\section{Research Approach}
%exploratory - justify!
I developed my research questions using a constructivist approach, where I sought to construct knowledge and theories, relying mainly on participants' views \cite{easterbrook2008selecting}. According to Easterbrook, a constructivist approach does not focus on verifying theories, but instead seeks to understand how people make sense of the world, and might build theories based on the context being studied. In this work, my goal was to seek student and instructor perspectives on the use of GitHub in an educational context. These perspectives would help determine whether or not `The GitHub Way' is a viable approach to education, and one that can influence the development of tools focused on computer science and software engineering education.

%more on social constructivism/constructivism? from student/teacher perspective?

I address the first research question with a thorough literature review, where I discuss work surrounding computer science and software engineering education, tools typically used in education in general, and tools specifically for educational use in these technical fields. This review also includes literature surrounding GitHub and similar Distributed Version Control Systems, particularly exploring their collaborative features and describing other instances in which they have been used in an educational context. In searching for literature, I focused on educational tools, exploring what researchers believe typical LMS lack and how those limitations can be addressed.

The second research question is addressed in two different studies---one involving instructors as participants, and another involving two courses as a case study. In both studies, the research methods used were congruent with exploratory research as we sought to explore the perspectives of instructors and students on the use of GitHub as a platform for education. In this thesis, both studies will be described in separate chapters, and the research methodology will be explained in more detail within each chapter.

\section{Thesis Organization}
The rest of this thesis is organized as follows.

Chapter 2 summarizes the related work, including literature on Computer Science and Software Engineering education, learning tools, and student engagement. The aim of this chapter is to elaborate on where this research fits into the surrounding literature.
%do we talk FOCUS GROUPS?!?!?!?!?!?!?!?

In Chapter 3, the `GitHub Way' is discussed further, where the features of GitHub are described in greater detail, including the possibilities for application to education. The aim of the chapter is to provide an overview of how GitHub might fit into the educational landscape.

In Chapter 4, I describe a study where instructors who were early adopters of using GitHub for education were interviewed. The aim of the study was to explore how and why educators used GitHub for their courses, including the benefits and challenges they experienced relating to the use of the tool.

%in which GitHub was used as a learning platform? as an LMS?
In Chapter 5, I describe a study conducted in which GitHub was used in two courses at the University of Victoria. The aim of the study was to explore the student perspective regarding what they believed to be the benefits and challenges of using such a platform for their courses. The study also aimed to identify the workflows students believed would best fit the use of a platform like GitHub.

In Chapter 6, I discuss the relevance of the work done and the implications on future learning tools. I also describe some recommendations for possible uses of GitHub in an educational context.

In Chapter 7, I outline possible future work and conclude the thesis by discussing how my research questions were addressed. Additional documents are found in the Appendices, including the interview questions for both studies.

%definitions?!?? is there a need?

%..DCVS and GitHub, use in Software Development world

%..CS/SE tools in education, what can be addressed
