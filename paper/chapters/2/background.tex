%Computer Science Education, why is collaboration/participation important?
%Social Aspect of Education
%The idea of the contributing student
%Tools in education
%Engaging students?
%Developments on tools

%what do I think are GH's advantages?
%transparency and awareness in group projects (is this just collaboration)
%contribution to class materials via PR - engagement?
%helping other projects (is this just collaboration)
%reusing and remixing others code or classes

\chapter{Background}
As computer science and software engineering evolves, so do the methods of educating students in those fields. Collaboration and developer contribution is often studied in current software engineering research as the `social programmer' becomes an increasingly common description of today’s developer \cite{treude2012programming}. Being a software developer in today’s landscape involves a social element, as developers often need to collaborate and coordinate with each other or contribute to the knowledge maintained by a community via social media. Recent research addresses collaboration as a common topic, evident in Whitehead's roadmap of collaboration in Software Engineering \cite{whitehead2007collaboration}.

An important goal of software engineering education is to provide students the skills needed to integrate theory and practice, preparing them for their professional careers. Students in this discipline need to be able to solve different problems both on their own and as members of a team. For these students, problem solving is best taught through examples and learned through practice \cite{jazayeri2004education}. Working in collaboration with other students on projects can, to an extent, simulate a real-world environment of working in a development team, and many university courses provide this by including group assignments.

As such, many have posited that collaboration is an important facet of software engineering that needs to be integrated into education. Included in Shaw’s roadmap for software engineering education is one such example of collaboration, where students need to study good examples of code and iterate on other people’s code \cite{shaw2000software}. Although this involves indirect collaboration, Shaw felt it was important to educate software engineers in such a way that they are able to read, understand, and build on other code. Jazayeri \cite{jazayeri2004education} developed a curriculum for his university, identifying a need to teach non-technical `soft' skills (such as communication and teamwork) alongside technical ones. In discussing the future trends of software engineering education, both Lethbridge \textit{et al.} \cite{lethbridge2007improving} and Mead \cite{mead2009software} highlight that teams are becoming increasingly distributed, and the social contact between group members can be very high. This suggests the importance of going beyond teaching just the technical skills, but to also account for the social nature of software engineering, where developers need to work together regularly in various ways. In doing so, students are provided experience in these real-world scenarios of collaborating in a team.

% This chapter highlights the literature surrounding tools used in post-secondary education, particularly those which enable collaboration for computer science and software engineering students.\todo{redundant...stuff like this is filler and doesn't provide value}
%satisfied with this opening for now; except this last paragraph

\textbf{Participatory Culture in Education} \\
Computer science and software engineering education has shifted away from the traditional method of learning in which material is transmitted unidirectionally by instructors and books to be absorbed by students. In 1998, Ben-Ari \cite{ben1998constructivism} discussed the concept of Constructivism, the theory that knowledge is actively constructed by students rather than passively absorbed through reading and lectures. Through Ben-Ari’s survey of constructivism in science and mathematics education, he argued that there needs to be more consideration for constructivist education in computer science.

Research then shifted towards the social construction model, which places a larger emphasis on the social and cultural context surrounding students. In this model, individual knowledge is created through interactions with others as well as interactions with the environment \cite{kim2001social}. This social emphasis is rooted in much of the research and theories in computer science and software engineering education. As an example, some researchers have explored communities of practice, which Wenger classifies as peers that share and develop knowledge in a common context \cite{wenger1998communities}, and whether or not these communities fit into an educational context \cite{ben2004situated}.
%check this wenger citation; also what is the point of this paragraph?

Jenkins \cite{jenkins2009confronting} discusses a similar concept---a participatory culture---as ideal in education, where students regularly contribute to each other's learning. A participatory culture is characterized by the following attributes:

\begin{itemize}
\item Relatively low barriers to artistic expression, where participants can create and remix content at will.
\item Strong support for creating and sharing one’s creations so they will be easily available to others.
\item Some type of informal membership, where more novice members can learn from the more experienced.
\item Members believe their contributions matter.
\item Members feel some degree of social connection with each other.
\end{itemize}

The concept of the participatory culture strongly reflects a social constructivist approach, as students contribute to each others' learning by interacting with the material and with others. This culture encourages students to remix and share content, give feedback, and help each other.

Specific to computer science education, Machanick \cite{machanick2007social} advocates a similar social constructivist model by making the learner more involved in knowledge creation and bringing them closer to experts. He proposes an action learning model where students formulate a model (the planning stage), attempt to carry it out (the action stage), and then reflect on what happened (the reflection stage) until a specific problem is solved. Social elements, such as reviewing other work or gathering advice from experts, would be included in each stage, and the learning would occur in the context of the group or with an ‘expert’ in the field. This model is one such example of researchers and educators attempting to integrate these social elements and a participatory culture into student coursework, exemplifying how the social construction model has permeated education. The next section discusses an approach that extends these social and participatory models based on student contributions. \\

\textbf{The Contributing Student in Computer Science} \\
Student engagement was a construct proposed by Astin \cite{astin1984student} as a developmental theory for college students surrounding the concept of involvement. This referred to ``the amount of physical and psychological energy that the student devotes to the academic experience''. According to Kuh \cite{kuh2001assessing}, this is an important concept due to the effects student engagement can have on student grades and student retention between first and second year. Moreover, some literature suggests that engagement can be achieved through group work and group learning \cite{bower2007groupwork}.

Collis and Moonen \cite{collis2001flexible} proposed a more social approach to education, the contributing student', where students contribute materials for other students to learn from. In this concept, the tool being utilized in the classroom plays an important role, as they note that the tool or site being contributed to should be largely empty before the learners and instructor fill it through course activities.

As this concept evolved, the key idea remained the same: learners should create or find learning materials and share them with others as a way to engage in their learning \cite{collis2006contributing}. By contributing to the course material with their findings and experiences, students can affect each others' learning. This means a student adopts several roles in a learning community, including being a co-creator of learning materials, being someone who extends the work of others (rather than just reading them), and being someone involved in self and peer evaluation.

In the computer science discipline, Hamer’s research group has conducted much of the research surrounding what he calls the ``Contributing Student Pedagogy'' (CSP), an extension of Collis’ approach. Hamer first reports his experiences \cite{hamer2006some} with this pedagogy in a number of courses, concluding that although not all students liked the approach at first, it helped students gain new perspectives and develop skills such as communication and teamwork. His group later formally defined CSP \cite{hamer2008contributing} as:

``A pedagogy that encourages students to contribute to the learning of others and to value the contributions of others.''

They observed various characteristics of CSP in practice: (a) the people involved (students and instructors) switch roles from passive to active, (b) there is a focus on student contribution, (c) the quality of contributions is assessed, (d) learning communities develop, and (e) student contributions are facilitated by technology. This pedagogy places an emphasis on the social interactions between both students and instructors, which is a key concept in much of the computer science and software engineering education literature described in this chapter. Falkner and Falkner \cite{falkner2012supporting} report on the effectiveness of CSP when they adopted it into their computer science curriculum, observing benefits such as increased engagement and participation, and the development of critical analysis, collaboration, and problem solving skills---important skills for a computer scientist.

In the literature surrounding the idea of ``the contributing student'', researchers emphasized the importance of the tools used in a course. Without the appropriate tools, according to Collis and Moonen \cite{collis2006contributing}, this approach to student engagement may not even be feasible in practice. In the next section, we explore the literature surrounding the tools often used in education, and how they fit the aforementioned social approaches to education. \\

\textbf{Tools and Supporting Technology in Education} \\
Regardless of the field, the use of software tools to support learning, teaching, material dissemination, and course management is an important aspect of education. Traditionally, university educators employ the use of learning management systems (LMS) to manage the courses they teach. LMS, such as Blackboard, Moodle, and Sakai, give instructors a variety of features for managing courses, such as file management, grade tracking, assignment hosting, and chat \cite{kumar2011comparative}. The use of an LMS provides students and educators with a set of tools for typical classroom processes, such as managing a student roster, forum discussions, or making announcements to the class.

With the rise of the social web and ‘Web 2.0’ technologies and services, as well as the increasingly social approach to education, the tools used for teaching and learning changed in a number of ways. Many have advocated leveraging these Web 2.0 technologies to support learning and teaching to create ‘e-learning 2.0’ \cite{downes2005feature}, where learning tools have transitioned into more social software, such as wikis and blogs. Social software can be defined as ``applications and services that facilitate collective action and social interaction online with rich exchange of multimedia information and evolution of aggregate knowledge'' \cite{parameswaran2007social}. These allow users to create content that is dynamic, remixable, and open to feedback. This transition to the use of social software, according to Downes \cite{downes2005feature}, involves a different type of distribution than traditional learning management systems where materials are not just disseminated, but also remixed and repurposed to involve more student participation, increasing engagement in their learning.

Researchers and educators in various fields have conducted a variety of studies using technology to increase student engagement and performance. In utilizing Twitter, Junco \textit{et al.} \cite{junco2011effect} increased student engagement and improved student grades by simply teaching students how they could utilize Twitter for their courses, such as by asking questions, continuing class discussions, and being given academic and personal support. Chen \textit{et al.} \cite{chen2010engaging} used data from the National Survey of Student Engagement (NSSE) to provide evidence for a positive relationship between using technology in learning and student engagement and desirable learning outcomes. Minocha \cite{minocha2009study} highlighted a number of case studies surrounding the use of social software to support student learning and engagement, seeing success in the use of virtual environment tools such as `Second Life', wikis, blogs, and Twitter.

Importantly, using these tools provides benefits beyond engaging students. In a study, Minocha and Thomas \cite{minocha2007collaborative} introduced wikis as an environment for their students in a software requirements course, where students would collaborate to create and discuss requirements together on a wiki. The instructors described the benefits of using such a system, where students enjoyed the feedback they received from other students and the ability to assess each others' work. Moreover, the instructors felt that wiki use was commonplace in industry, and therefore, it was beneficial for the students to develop their communication and teamwork skills, which are transferable to industry.
%Minocha - A case study-based investigation of students’ experi- ences with social software tools????

In many cases, `2.0 technologies' were simply used in conjunction with LMS with great success, with an example leading to an increase in grades \cite{conde2014evolving}. Students have come to value the way such tools provide convenience, and even expect these tools to be available for communication purposes \cite{caruso2007ecar}. As a result, developers began incorporating these technologies into Learning Management Systems to account for more social approaches. Edrees, for example, \cite{edrees2013elearning}, compares the `2.0' tools and features of Moodle and Blackboard, two of the more popular LMS, identifying that they both added features such as wikis, blogs, RSS, podcasts, bookmarking, and virtual environments to become more social.

However, despite the increase of social features in LMS, many researchers and educators have expressed concerns regarding their readiness to incorporate student participation. McLoughlin \cite{mcloughlin2007social} believes that participatory learning lends itself well to education as students are provided with more learning opportunities where they can connect and learn from each other. However, he notes that LMS tend to be more administration-focused, and that there were signs that Web 2.0 tools could make learning environments more personal, participatory, and collaborative. Similarly, Dalsgaard \cite{dalsgaard2006social} believes that LMS should only hold a minor role compared to separate, more social tools. He argues that students should be provided with a myriad of tools for independent work, reflection, construction, and collaboration. He does, however, acknowledge that effort needs to be made so that social tools like blogs and wikis can support educational activities, because they are otherwise not educationally-focused tools.

Further weaknesses of LMS are outlined by Mott \cite{mott2010envisioning}, who believes that LMS impede teaching and learning innovation because courses often expire after some time, disrupting continuity. As well, he argues that they offer few opportunities for student-initiated learning and that courses are `walled gardens'---closed off to outsiders---that limit the potential for collaboration and the remixing of work. García-Peñalvo \cite{garcia2011opening} believes that students need to be placed at the centre of the e-learning process, but that the current generation of LMS are insufficient for this due to their lack of openness, resistance to change, lack of integration with informal context, and so on. These criticisms of the traditional LMS suggest a need for change, and the next section describes the steps being made to make tools that are more focused on student participation and contribution, particularly in computer science and software engineering.

\textbf{Evolving Learning Tools for Computer Science} \\

Researchers have attempted to conceptualize or build enhancements for LMS to address weaknesses and further focus them towards computer science education. Rossling \textit{et al.} \cite{rossling2008enhancing} introduced the concept of CALMS, a Computer Augmented Learning Management System, wherein the typical LMS would be extended to support activities such as allowing students to assess each other, automatically grading programming submissions, and actually programming through a connected IDE. The majority of the features described in their idea of CALMS provide students with various ways to contribute to each others' learning. The same working group investigated the open source LMS Moodle \cite{rossling2010adapting}, identifying plug-ins that can augment the system to support features like inclusion of source code, shared calendars for groups, automated assessment of programming assignments, and other features more focused for computer science and programming work rather than student participation.

Other attempts have been made to leverage existing tools to benefit computer science and software engineering students. Team projects have implemented collaboration tools such as the Jazz plug-in for the Eclipse IDE, with its version control, wiki, and instant messaging features \cite{meneely2009preparing}, giving students more ways to interact with the material and with each other. Reid \& Wilson \cite{reid2007drproject}  created the DrProject portal, focusing on developing social media-type features (wiki, mailing list, tags) around a coding repository (Subversion). Educators are seeing the advantages of using tools that software engineers and developers use in the real world as students are familiarized with the tools prior to starting their careers. Moreover, in many of these cases, the concepts or developments made to these tools attempt to further develop the social element of the tools used in the classroom, lending credence to the growing need for tools to support these social activities.

In 2011, Hamer \cite{hamer2011tools} reviewed tools that support CSP in computer science education, noting eight main tool characteristics:
\begin{itemize}
\item The nature of the interactions between students, as facilitated by the tool.
\item The number of students taking part in the sharing activity.
\item The nature of the artifacts created.
\item The extent to which the use of the tool has been evaluated.
\item Feedback mechanisms from students to the creator of the artifacts.
\item Gathering data on student activity.
\item The extent to which the tool facilitates the entire CSP activity.
\end{itemize}

%you are here
While Hamer's literature search provided a number of tools that meet many of these characteristics, they were surprised that there weren't more examples of tools that support student-contributed learning activities. As well, they reported that many of the tools seemed to only be used within the institution where they were developed, not supporting cross-institutional use. This suggests that tools in the computer science and software engineering disciplines need improvements to further support student participation and collaboration.

As distributed version control systems play a crucial role in many software projects, including their support for developer contribution and collaboration, researchers have attempted to see how these systems can benefit education. Reid \& Wilson \cite{reid2005learning}, introduced Concurrent Version Systems (CVS) for their classes, making it easier for students to work in groups as well as providing a history of student work. Beyond those obvious advantages, instructors and teaching assistants were also able to assist students better as they could easily retrieve an up-to-date copy of student work. Similar advantages are found when other version control tools such as Subversion \cite{clifton2007subverting} and Git \cite{griffin2013github} are used in education, using features such as branching and merging to better organize assignments and assignment submission.

Git is one of the more popular distributed version control systems used for today's software projects, and a particular way of interfacing with Git is through the use of the GitHub or Bitbucket Web platforms that include collaborative features such as wikis, issue trackers, and tagging. Student projects could leverage the issue tracker on each tool so that issues, comments, and responses can be seen by all \cite{kelleher2014employing}. Haaranen & Lehtinen \cite{haaranen2015teaching} provide an example of Git being utilized in a large-scale (200 student) computer science classroom through the GitLab Web portal, citing benefits such as the ability to correct course material, and the experience provided for students in using a tool relevant to the industry as a whole.

% \todo{sum up something here}

%King and Robinson \cite{king2009pretty} used electronic voting systems to increase the in-class participation of students - though they found no correlation between the use of these systems and student grades.

% \section{Transparency \& Awareness?}

% REMIXING
% TRANSPARENCY AND AWARENSS IN EDUCATION, AND IN DEV

%misc papers
%[PDF] Collaborative learning: Some possibilities and limitations for students and teachers
%M Bower, D Richards - 2006 - researchonline.mq.edu.au
%http://www.researchonline.mq.edu.au/vital/access/services/Download/mq:10990/DS01
% This was a highly significant difference between the observed and desired amount of time spent working with others
