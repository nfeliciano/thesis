\chapter{Background}

The landscape of educating computer scientists and software engineers sees constant change as software engineering as a field evolves. Much of current research involves studying collaboration in software engineering as the ‘Social Programmer’ becomes an increasingly common description of today’s programmer [cite Treude 2012]. Being a programmer or software developer in today’s landscape involves a big social element from needing to collaborate with others, needing to coordinate with others, and even finding and sharing information online on social media. We can see recent research addressing this as they study collaboration processes and tools for software engineers [cite Whitehead 2007].

An important goal of software engineering education is to provide students the skills to integrate theory and practices, preparing them for their professional careers. As the real world requires it, students in this discipline need to be able to solve different problems both on their own and as members of a development team. Problem solving in software engineering is best taught through examples, and learned through practice [cite Jazayeri 2004]. Working in collaboration with other students on projects can more or less simulate a real-world environment of working in a development team, and many university courses provide this by including group project components for students.

As such, many have posited that collaboration is an important facet of software engineering that needs to be integrated into education. Included in Shaw’s roadmap for software engineering education was the need to study good examples of code and iterate on other people’s code [cite Shaw 1999]. Although this involves indirect collaboration, she felt it was important to educate software engineers in such a way that they are able to read, understand, and build on other code. Jazayeri [cite 2004] developed a curriculum for his university, identifying a need to teach the non-technical skills (communication and teamwork) alongside the technical ones. Both Lethbridge et al. [cite 2007] and Mead [cite 2009], in discussing the future trends of software engineering education, highlight the landscape in teams are increasingly becoming distributed, and the social contact can be very high. This all suggests the importance of going beyond teaching just the technical skills, but to also account for the social nature of software engineering, where developers need to work together regularly in various ways. In doing so, students are provided experience in these real-world scenarios of collaborating in a team.

\section{The Social Aspect}
The landscape of computer science and software engineering education has shifted from the more traditional way of learning by absorbing material from books and lectures. Ben-Ari [cite 2001] discusses the concept of Constructivism, which is the theory that knowledge is actively constructed by students rather than passively absorbed through reading and lectures. Through Ben-Ari’s survey of constructivism in science and mathematics education, he argues that there needs to be more consideration for constructivist education in computer science.

More recently, research has shifted towards the social construction model, which places a larger emphasis on the social and cultural context, and individual knowledge is created through interactions with others and the environment [cite Kim 2001]. This social emphasis is rooted in much of research on and many of the theories in computer science and software engineering education. There has been research on communities of practice, which are essentially communities of peers that share and develop knowledge in a common context {rw} [cite Wenger 1998], and how they fit into an educational context [cite Ben-Ari 2004].
Jenkins [cite 2006] discusses a similar concept - the participatory culture - as an ideal in education, where students are part of the course material. A participatory culture has the following attributes:

Relatively low barriers to artistic expression, where participants can create and remix content at will

Strong support for creating and sharing one’s creations so they will be easily available to others

Some type of informal membership, where more novice members can learn from the more experienced

Members believe their contributions matter, and feel some degree of social connection with each other

The concept of the participatory culture strongly reflects a social constructivist approach, as students take a greater part in the system {ch}. This culture allows students to easily remix and share content, give others feedback on their content, and help each other out when needed.

With regards to computer science education, Machanick [cite 2007] advocates a more social constructivist model by making the learner more involved in knowledge creation and bringing them closer to experts. He proposes an action learning model, wherein students would learn by formulating a model (the planning stage), attempt to carry it out (the action stage), and then think about what happened (the reflection stage) until a specific problem is solved. The social elements would be added in all stages and the learning would be done in the context of the group, or with an ‘expert’ in the field. {eg, gi}

\section{The Contributing Student}

Collis [cite 2005] proposes a more social approach to education he coins “The Contributing Student”. He notes that there are necessary attributes for functioning productively in an era where one is constantly needing to gain knowledge:

Continuously updating and changing skills

Using electronic networks effectively and efficiently

Handling the mobility of services, information, workforce

Working in multi-disciplinary and global teams

Deriving local value from global systems

Acting autonomously and reflectively, in socially heterogeneous settings

The important aspect of Collis’ pedagogy is for learners to create learning materials and share them with others {rw}. In this approach, students affect others’ learning by contributing to the learning resources with their own knowledge and experiences, as well as experiences and materials found on the web. This means a student adopts several roles in a learning group or community, including being a co-creator of learning materials, being someone who extends the work of others rather than just reading them, and being someone involved in self and peer evaluation. This mirrors the social constructivist learning model where students are involved in the construction of each other’s knowledge.

In the computer science discipline, Hamer’s research group has conducted much of the research surrounding what he calls the “Contributing Student Pedagogy” (CSP), an extension of Collis’ approach. Hamer first reports his experiences [cite 2006] with this pedagogy in a number of courses, concluding that although not all students liked the approach at first, it helped develop a number of skills such as communication, teamwork, and getting a new perspective. His group later defines a CSP [cite Hamer et al. 2008] as:

A pedagogy that encourages students to contribute to the learning of others and to value the contributions of others.

This definition includes two important aspects: that students contribute to the learning of others and that they value the contribution of others. They observe various characteristics of CSP in practice: that the people involved (students and instructors) switch roles from passive to active, that there is a focus on student contribution, that the quality of contributions is assessed, that learning communities develop, and that student contributions are facilitated by technology. This pedagogy overall places emphasis on the social interactions between both students and instructors, epitomizing the social constructivist approach that much of research on computer science education has deemed important {rw}. Falkner and Falkner [cite 2012] reports on the effectiveness of CSP when they adopted it into their computer science curriculum, observing benefits such as increased engagement and participation and the development of critical analysis, collaboration, and problem solving skills – important skills for a computer scientist.

\section{Tools and Supporting Technology}

Of course, an important aspect of computer science education is the use of software to support learning, teaching, and the management of materials and courses {rw}. Traditionally, in many disciplines including computer science and engineering, educators in universities employ the use of learning management systems (LMS) to manage the courses they teach. LMS, such as Blackboard, Moodle, and Sakai, give instructors features for managing courses such as file management, grade tracking, assignment hosting, chat, and other features [cite Kumar 2011]. The use of an LMS provides students and educators with a set of tools for typical classroom processes (in face-to-face classes, distance classes, and blended classes) such as managing a student roster, forum discussions, or making announcements to the class.

With the rise of the social web and ‘Web 2.0’ technologies and services, the tools used for teaching and learning changed in a number of ways. Many have advocated leveraging these Web 2.0 technologies to support learning and teaching to create ‘e-learning 2.0’ [cite Downes 2005], where the learning tools transitioned into more social software such as wikis and blogs. Social software can be defined as “applications and services that facilitate collective action and social interaction online with rich exchange of multimedia information and evolution of aggregate knowledge” [Parameswaran & Whinston 2007]. These allow users to create content that is dynamic, remixable, and open to feedback. This transition to the use of social software, according to Downes, involves a different type of distribution than traditional learning management systems where materials are not just disseminated {ch}, but also remixed and repurposed to involve more student participation.

Educators in all fields have been leveraging social software in a myriad of ways, using social network services such as Facebook and Twitter [cite Goh 2013] and blogs {eg} to make their classrooms more social and more engaging. Students have come to expect web and collaborative technologies to be integrated into their face-to-face classes, believing that these tools make the educational experience more convenient and effective [cite Salaway & Caruso 2007]. Indeed, there seems to be a positive correlation between the use of web-based learning technologies and student engagement, as students who utilize such tools tend to score higher in traditional engagement measures such as collaborative learning and student-teacher interaction [cite Chen et al. 2010]. As such, learning management systems began incorporating these technologies into their platforms to account for more social approaches. Edrees [cite 2013] compares the ‘2.0’ tools and features of Moodle and Blackboard, two of the more popular LMS, identifying that they’ve both added features such as wikis, blogs, RSS, podcasts, bookmarking, and virtual environments to become more socialaaa.

However, despite the increase of social features in LMS, many researchers and educators have expressed concerns regarding their readiness for student participation. McLoughlin [cite 2007] believes that participatory learning lends itself well to education as students are provided with more learning opportunities, where they can connect and learn from each other. However, he notes that LMS tend to be more administration-focused, though there were signs that Web 2.0 tools were resulting in learning environments that were more personal, participatory, and collaborative. Dalsgaard [2006], who similarly holds a social constructivist approach to learning, believes that LMS should only hold a minor role compared to separate, social tools. He argues that students should be provided with a myriad of tools for independent work, reflection, construction, and collaboration, acknowledging that effort needs to be made so that social tools like blogs and wikis to support educational activities, as they aren't otherwise focused on that aspect {rw}.

Further weaknesses of LMS are outlined by Mott [2010], who believes that LMS impedes teaching and learning innovation because courses typically expire after some time which disrupts continuity. As well, he argues that they offer few opportunities for student-initiated learning and that courses are ‘walled gardens’, closed off to outsiders and therefore limiting the potential for collaboration and remixing work. García-Peñalvo [2011] believes that students need to be placed at the centre of the e-learning process, but that the current generation of LMS is insufficient for this due to their lack of openness, resistance to change, lack of integration with informal context, and so on.

\section{Evolving the Learning Tool}

Researchers have attempted to conceptualize or build enhancements for LMS to address those weaknesses and further focus them towards computer science education. Rossling et al. [cite 2008] introduce the concept of CALMS, a Computer Augmented Learning Management System, wherein the typical LMS would be extended to support various activities such as allowing students to assess each other, automatically grading programming submissions, and actually programming through a connected IDE. The same working group specifically investigates open source LMS Moodle [cite 2010], identifying plugins that can augment the system to support features like inclusion of source code, shared calendars for groups, automated assessment of programming assignments, etc.

Other attempts have been made to leverage existing tools to benefit computer science and software education students. Collaboration tools such as the Jazz plugin for the Eclipse IDE can be used for team projects in computer science for its version control, wiki, and instant messaging features [cite Meneely & Williams 2009]. Reid & Wilson [cite 2007] develop the DrProject portal, focusing on developing social media-type features around a coding repository (Subversion) such as a wiki, a mailing list, and tags. Educators are seeing the advantages of using tools that software engineers and developers in the real world use, and this is beneficial for making students more familiar with the tools as well as (in most cases) bringing in the social element to the tools used in the classroom.

In 2011, Hamer [cite] reviews the tools that support CSP in computer science education, noting eight main characteristics of these tools:

The nature of interaction between students, as facilitated by the tool

The number of students taking part in the sharing activity

The nature of artifact created

The extent to which the use of the tool has been evaluated

Feedback mechanisms from students to the creator of the artifact

Gathering data on student activity

The extent to which the tool facilitates the entire CSP activity

Whether the tool is tied to the teaching of a single topic, or has wider use
While their literature search provided a number of tools that meet many of these characteristics, they were overall surprised that there weren’t more examples of tool-support for student-contributed learning activities. As well, they reported that many of the tools appeared to only be used within the institution where they were developed, not supporting cross-institutional use. This suggests that tools in the tools in the computer science and software engineering disciplines need improvements to further support student participation and collaboration.

As distributed version control systems play a crucial role in many software projects, researchers have attempted to see these systems’ benefits for education. As they have support for developer contributions and collaboration, these systems could potentially do the same for education. Reid & Wilson [cite 2005], in introducing Concurrent Version Systems (CVS) for their classes, would make it easier for students to work in groups as well as providing a history of student work. Beyond those obvious advantages, instructors and teaching assistants were also able to assist students better as they could easily retrieve an up-to-date copy of student work. Similar advantages are found in other version control and distributed version control tools such as Subversion [cite Clifton et al. 2007] and Git [cite Griffin & Seals 2013], using features such as branching and merging to better organize assignments and assignment submission.

Git is one of the more popular distributed version control systems used for software projects today, and a particular way of interfacing with Git is through the use of GitHub or Bitbucket, web platforms that include many collaborative features such as wikis, issue trackers, and tagging. Student projects could leverage the issue tracker on each tool so that issues, comments, and responses can be seen by all [cite Kelleher 2014]. As well, the news feed and the ability to summarize contributor activity on GitHub can be useful for both students and instructors, the former for keeping updated with the relevant repositories and the latter for grading students based on activity [cite Zagalsky et al. 2014].



% REMIXING
% TRANSPARENCY AND AWARENSS IN EDUCATION, AND IN DEV
