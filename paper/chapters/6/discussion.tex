\chapter{Discussion}
The studies in the two previous chapters explored the use of GitHub as an educational tool, first from the instructor's perspective and second from the student's. In this chapter, we discuss the viability of using GitHub and the GitHub Way in courses, a number of recommendations for instructors who wish to use GitHub in their courses, and the implications of this research.

\section{The Viability of GitHub for Education}
In this thesis, we have discussed how GitHub can be used to support teaching and learning, as well as why instructors might consider using GitHub for their courses. With multiple Learning Management Systems available for higher education institutions to consider, it is compelling to ask: can GitHub on its own be one such tool, or does its use in education require another, administratively-focused tool to be used in conjunction?

% It is important to make a distinction here - discussions surrounding GitHub's viability for education applies to not just GitHub, but to similar DVCS tools that offer similar features such as BitBucket and Mercurial.

%yes, but not GitHub
Investigating Malikowski \textit{et al.}'s \cite{malikowski2007model} model, LMS activities are split into five categories:
\begin{itemize}
\item \emph{Transmitting Course Content}, which instructors most often use LMSs for. GitHub not only supports activities related to this category, but it can also extend it by providing a two-way transmission wherein students can also easily contribute to the course content.
\item \emph{Creating Class Interactions}, which GitHub supports through the commenting system, either in-code or in the `issues' pane. These interactions are mostly asynchronous, however, as GitHub provides no support for interactions similar to a chat environment, though external chat tools such as Slack\footnote{\url{https://slack.com}} and Gitter\footnote{\url{https://gitter.im}} can integrate GitHub to display the activities in a repository. Student interactions on GitHub can extend to reviewing and contributing to each other's work.
\item \emph{Evaluating Students}, which GitHub enables by allowing instructors to comment on student work, which notifies students upon receiving comments. One important limitation in this category, however, is that in the typical case, data stored in GitHub's repositories will be stored on GitHub servers in the United States, rather than in something an institution can have control over.
\item \emph{Evaluating Courses and Instructors} is an activity GitHub does not natively support. An instructor would need to use external tools such as survey generators.
\item \emph{Creating Computer-Based Instruction} is an activity that instructors could use GitHub for with some work. A basic form of computer-based instruction is online quiz generation, a feature built into many modern LMSs, but a feature GitHub does not support without building an external tool. However, automatic grading tools can be built and utilized whenever a student makes a push, creating a form of computer-based instruction.
\end{itemize}

As such, GitHub meets most of the basic features traditional LMSs often include. However, with data typically being stored on GitHub's servers, many institutions require much of the administrative artifacts such as class rosters and grading to be under their control, which means that another tool will need to be used in conjunction with GitHub. Fortunately, there are solutions, as outlined below.

Some of the instructors we interviewed were able to set up servers to be used for GitHub in their own space, thereby having control of the data. This can be done through GitHub Enterprise\footnote{\url{https://enterprise.github.com/home}}. Another solution would be to set up a tool that is similar to GitHub that can use servers that are independently set up, such as GitLab\footnote{\url{https://about.gitlab.com}}. These solutions would eliminate the limitation of having external servers, therefore enabling the hosting of secure artifacts. The disadvantage with these approaches, however, is that they minimize the benefits of having a course visible and open for outside communities to participate in.

%say that I recommend using it somewhere here?
Overall, using GitHub for educational purposes can be an effective alternative for instructors who wish to avoid the limitations of traditional LMSs \cite{garcia2011opening}. Beyond using GitHub as an LMS, however, GitHub offers many advantages and opportunities not present in traditional LMSs. Therefore, it should be a tool that instructors, particularly those teaching computer science and software engineering, should consider using to support their courses. What GitHub offers are unique ways of engaging students by providing them opportunities to contribute to the course and to each other's learning. These opportunities to engage students build a culture where students can perform better as a result \cite{kuh2001assessing}. For instructors, GitHub also offers unique opportunities for grading and for easy ways of reusing and remixing course material from previous and for future iterations of the course.

\section{Recommendations for Educators}
This section provides recommendations for educators who want to use GitHub to support their courses. These recommendations are based on the findings from the two studies presented in this thesis, as well as from the review of literature surrounding tools in computer science and software engineering education.

Before proceeding, I note that GitHub has their own set of recommendations for setting up an organization for a class\footnote{\url{https://education.github.com/guide}}. Their classroom guide is useful for those looking for a step-by-step process, where they recommend applying for an organization for a course and assigning a private repository for each assignment for each student. Likewise, it can also be helpful to use the available resources: use GitHub support, look for other instructor experiences for guidance, or discuss experiences in a blog or in spaces dedicated to the topic\footnote{\url{https://github.com/education/teachers/issues}}. Contributing to these resources can serve towards building a common knowledge base for instructors to share to and learn from. \\

\textbf{Recommendation: Use GitHub's Features} \\
Computer science and software engineering students benefit from early exposure to Git and GitHub. By utilizing these (or similar) tools in their courses, educators provide students a way to familiarize themselves and practice with these tools, which can benefit their careers. Beyond exposure, hosting assignments, projects, and code on student accounts could be valuable when seeking employment, as companies continue to investigate the online presence prospective employees have (e.g., their GitHub accounts) for hiring purposes.

While simply using GitHub as a system for material dissemination can be helpful, using more of GitHub's features, such as pull requests and issues, provides even more benefits for the students. For example, allowing students to contribute to the course and to each other's work can help develop skills such as teamwork and communication \cite{hamer2006some}. As another example, exposure to GitHub's Issues feature, even for basic discussions, was helpful for one of the students interviewed in the second study as the student learned how the feature works for use in future projects.

Educators can furthermore use GitHub's transparency features to provide feedback to students in unique ways. For example, instructors can trace the history of student projects and assignments hosted on GitHub, and instructors can detail where students made mistakes and can intervene when a student seems to be struggling. Moreover, in group projects, instructors can note how much work each student has contributed, and can use this transparency for assigning grades.

One important lesson noted from the case study was to communicate the workflow the instructor decides clearly and properly to the teaching team and to the students. When deciding to use a feature like pull requests on course material, for example, the instructor must advertise this workflow properly, perhaps even offering bonus points for added material. To communicate a workflow to students and introduce GitHub and its features to novices, instructors should consider creating a guide or hosting a tutorial session. \\

%notifications

\textbf{Recommendation: Use Free Private Repositories for Single Solution Assignments} \\
%type of course - better for open-ended
%assignment submission
Many students and instructors believed that GitHub worked best when a course has open-ended projects and assignments. This belief is because of the plaigarism concerns that exist when students are putting their code up online where others can potentially see their solutions. Of course, students can host their code in private repositories controlled by the instructor; if the instructor creates a private repository for each student to submit their assignments and adds only the student as a collaborator, plaigarism would only be as much of a concern as it would be without using GitHub.

This style of repository management (where a private repository is dedicated to each student) could work for assignment submission as well. The instructor could ask the students to create a branch, or ask the students to fork off the main repositories and make the forks private, and then mandate that the student must make a pull request before a deadline. Thanks to GitHub's transparency features, an instructor can continuously observe the work in each student's repository and can provide further assistance to students based on the work history.

However, the set up for this more private style of repository management requires some time and assistance from GitHub. There are two options: first, students can apply for student accounts which grants them 10 free private repositories, one of which can be used for the course in question. That is, however, a time consuming process as students must wait for GitHub to approve their request. The second option is to create an organization for the course, which is granted an amount of private repositories depending on how much the instructor pays. While GitHub has stated that they would give teachers a free organization for their courses\footnote{\url{https://github.com/blog/1775-github-goes-to-school}}, an organization must be set up well before the course begins in order to get the private repositories in time.

Moreover, if assignments are in private repositories and are single solution assignments, you limit one of the most important benefits of using a system like GitHub---the ability to view, comment on, and contribute to the work of other students. As such, although GitHub is usable and helpful in any type of course, courses with open-ended projects and courses with a culture of participation are where instructors and students will see the primary benefits of using GitHub as a learning tool. If an instructor chooses to pursue the open-ended style of work similar to the courses in this study, it is recommended that they list projects and assignments on the home page using the readme markdown file so students can easily access the other projects.

That said, GitHub continues to offer its benefits when used to submit single solution assignments. It involves some preparation to get free private repositories for students, but at the same time, it allows instructors to provide better feedback through versioning, and it maintains the benefits for students of learning Git and GitHub and hosting their work for future portfolio use (if allowed to publicize their work after the course concludes). \\

\textbf{Recommendation: Encourage Contributions from the Students} \\
%contributions from others (slides in html, comment on other projects issues)
Another way to utilize GitHub is to encourage contribution from the students in the ways that GitHub affords them. First, students can contribute to the course materials by making corrections, changes, and adding resources. Second, students can contribute to other students' work and projects (provided the work is open-ended). And third, students can contribute to projects outside the course by making changes and pull requests in open-source repositories. Encouraging this `Contributing Student Pedagogy' can help students develop skills such as critical analysis and collaboration \cite{falkner2012supporting}.

Moreover, all of the student contributions are potentially visible for the course instructor to see. For example, if contributing to other students' projects, they could simply raise an issue in their repository, or go as far as to making the changes themselves and creating a pull request from it, both of which are visible to the course instructors. One issue with student contributions that must be noted is that contributing to the course materials could present difficulties depending on the file types used, as binary files such as PDF documents and PowerPoint slides are not compatible with Git's versioning. For this reason, I recommend hosting class material and slides in either markdown or HTML, file types that GitHub supports and can be easily altered using its Web platform.

\section{Implications for the Future}
Another important consideration from this work relates to the future of tools for computer science and software engineering education---what's next? First, we consider the importance of participation, group work, and group learning for students in technical fields in order to develop non-technical `soft' skills such as communication and teamwork \cite{jazayeri2004education}. The two studies in this thesis demonstrate how using GitHub can unlock activities where students can contribute to each other's learning, and as a result, I believe that the GitHub Way can be a beneficial addition to current and future learning tools.

%GitHub for Education
The fact that GitHub easily supports participatory activities has multiple implications. Literature has shown that LMSs have been adding `Web 2.0' features such as blogs and wikis to their feature set \cite{downes2005feature}---students are being offered more opportunities to participate by discussion or by contributing content in blogs or wikis. Where the GitHub Way excels in education, however, is in the opportunities for students to contribute to and change the materials, and to contribute to each other's learning by getting involved in and providing feedback to projects other than their own. This is potentially the next step for Learning Management Systems, where students are more easily able to make these contributions to the work of others. The concern, however, is that similar to how LMSs have implemented 2.0 technologies, implementing features similar to GitHub in an LMS might seem forced and haphazardly planned.

As such, another possible path is the `GitHub for Education' Greg Wilson discussed\footnote{\url{http://software-carpentry.org/blog/2011/12/fork-merge-and-share.html}}, where a tool like GitHub can be altered or built to be more focused towards education. The main weakness of GitHub when used in this context is in the lack of flexibility in its privacy and in the lack of administrative functions such as gradebooks and announcements. Meanwhile, there are open-source alternatives to GitHub such as GitLab\footnote{\url{https://about.gitlab.com}}, that could be further developed into a tool that fulfils more educational needs. As an example, it could be valuable to implement a form of announcements, a notification feature that students have more control over, and a way to make some discussions or issues within a repository private while others remain public.

In summary, this work has shown the viability of using GitHub for education, and has demonstrated why the GitHub way should be considered when deciding which tools to use to support a course. Based on the findings of this work, I included a set of recommendations for educators interested in using GitHub as a learning tool, and list the implications on tools that could provide the same benefits as GitHub while mitigating the limitations. In the next chapter, I provide concluding remarks as well as some possible directions for researchers to take this work in the future.

%probably in conclusion
%\section{Limitations}

%\section{Future Work}
