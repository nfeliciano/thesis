\chapter{The Instructor Perspective}
This chapter highlights the first study conducted regarding GitHub use in education. The study was conducted in collaboration with Alexey Zagalsky, Dr. Margaret-Anne Storey, Evian, and Will. My personal contributions include being a part of the data collection and interviewing educators, the data analysis being involved in the coding process, and summarizing the findings and the discussion in collaboration with Alexey Zagalsky and Dr. Storey. %need I be more specific?

\section{Motivations}
This study began as a course project for three of the authors, Alexey, Will, and Evian, as a part of a course taught by Dr. Storey where I served as a teaching assistant. The authors first noted the popularity of GitHub and how it was seeing widespread adoption in other areas\footnote{\url{http://www.wired.com/2013/09/github-for-anything/}}, transforming how people collaborate over a shared repository [cite]. One of GitHub's main strengths is in the awareness and transparency features it provides to team, project and community members [cite]. These features positively influence how people contribute to projects [cite].

%\cite{begel2013social}
%\cite{Dabbish:2012:SCG:2145204.2145396}
%\cite{Tsay:2012:SMS:2141512.2141583}

Only a few years after GitHub's 2007 release, well-known computer science educator Greg Wilson suggested\footnote{\url{http://software-carpentry.org/blog/2011/12/fork-merge-and-share.html}} that GitHub could be used for learning materials despite some limitations: \begin{quote}\textit{Would it be possible to create a ``GitHub for education?'' Right now, I think the answer is ``no'', because today's learning content formats make \textbf{merging hard}. Whatever a ``GitHub for education'' would look like, it would not be yet another repository of open learning materials. There are lots of those already, but almost all their content is write-once-and-upload, ... rather than \textbf{sharing course content in a reusable, remixable way}}.\end{quote}
In 2012, he further elaborated\footnote{\url{http://software-carpentry.org/blog/2012/04/github-for-education.html}}: \begin{quote}\textit{``GitHub for Education'' isn't necessarily, ``Let's put educational materials in GitHub'', but rather, ``Let's \textbf{facilitate a culture of spontaneous-but-structured collaboration} and improvement.''}\end{quote} He recognized that the majority of learning management systems introduced friction for instructors trying to reuse and share course materials---the type of problem that the software development community solved through the use of tools such as GitHub \cite{Dabbish:2012:SCG:2145204.2145396}.

In an effort to promote GitHub to higher education, GitHub launched the GitHub Education Website\footnote{\url{https://education.github.com/}} in 2014.
In response to this and other \textit{ad hoc} opinions on the benefits of using GitHub for education, they began to conduct a study to examine whether and how GitHub, with its powerful collaboration, social and awareness features, can be used for educational purposes.

The result was a study conducted with 5 participants and some early findings and written into a paper for the course. As a personal interest, I joined the group to continue the study to produce a conference paper, published in ACM's Conference on Computer-Supported Collaborative Work (CSCW) 2015.

\section{Research Questions}
We devised a number of research questions geared towards exploring the niche educational use of GitHub, a tool otherwise geared towards software developers. The research questions addressed in this study include:

\textbf{How does GitHub support learning and teaching?} We investigate how GitHub is used and for what purposes within the education domain. %Our findings indicate that even though GitHub use in education mirrors the way traditional learning systems are used, the implications differ substantially.

\textbf{What are the motivations and benefits of using GitHub for education?} Based on testimonies from our study participants, we explore the motivations for GitHub use in education and the possible benefits it might bring to support learning. We also look at the specific features of GitHub that are being used to support learning.

\textbf{What challenges are related to the use of GitHub for education?} And finally, we examine the challenges educators and their students face when using GitHub to support learning and teaching. We provide specific examples based on interviews with educators, and later synthesize recommendations for educators wishing to use GitHub, as well as recommendations to the GitHub design team.

\section{Research Design}
In the first phase of our study, we used online research methods \cite{wakeford2008fieldnotes} and searched for resources (such as blog posts) that described the personal experiences of educators using GitHub to support learning or teaching. The results from this phase are presented in the Background section.

Through this exploratory phase, we found creative and successful examples of GitHub supporting teaching and learning in the classroom. We also found discussions on the challenges and difficulties of using GitHub. This allowed us to refine our research questions and guided us in the following phases of the study, such as shaping the questions to ask during interviews.

In the second phase of our study, we interviewed 15 participants, including one of the blog authors from the first phase. In this phase, we were able to thoroughly investigate the usefulness and potential of GitHub in education. Through iterative analysis of the data collected, several themes emerged around the motivations for and challenges of using GitHub to support learning.

These themes informed the third phase of our research: a follow-up survey sent to interviewees from the previous phase and to other educators using GitHub for education. The goal of this survey was to receive interviewee feedback on our interpretations, but also to gain additional perspectives from other educators that use GitHub.

\subsection{Participants}

\subsection{Data Collection}

\subsection{Data Analysis}

\section{Findings}

\section{Discussion?}
